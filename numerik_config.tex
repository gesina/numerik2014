% % % % % % % % % % % % % % 
%
%     Skript zu NUMERIK I
%           WS14/15
%           von Prof. Dr. Blank
% Universität Regensburg
%
%
%	Config Datei: Definitionen etc.
%
% % % % % % % % % % % % % %


% ESSENTIAL PACKAGES
% ---------------------------------------------------------------
%% encodings
\usepackage[T1]{fontenc}      %font-encoding WITH e.g. ö (default is OT1 with 7- instead of 8-bit)
\usepackage[utf8]{inputenc}   %input-encoding WITH e.g. ö (depends on system), better after fontenc
% ..................................................
%% language
\usepackage{babel}	       %rules typical for chosen language(s)
% ..................................................
%% font-settings
\usepackage{lmodern}	       %font: latin modern
\usepackage{microtype}	       %micro-typographic optimizing, e.g. ligatures
% 
% EXTRA PACKAGES
% ---------------------------------------------------------------
%% pagestyle
\usepackage{scrpage2}		%KOMA pagestyles scrheadings, scrplain
% \usepackage{enumitem}		% numeration items like \alpha*)
\usepackage{enumerate}		% easy numeration
% ..................................................
%% quotes
\usepackage{csquotes}		%easy quotes with \enquote{...}
% ..................................................
%% maths
\usepackage{amsmath}		%improves maths-sections
\usepackage{mathtools}          %provides tools like \DeclareParedDelimiter or \DefineMathOperator
\usepackage{amsthm}		%for proofs etc.
\usepackage{amssymb}		%further maths symbols like \square for proofs
\usepackage{dsfont}		%for \mathds{letter} to create e.g. the rational numbers symbol Q
\usepackage{stmaryrd}		%for math symbols like lightning bold
\usepackage{MnSymbol}
% \usepackage{bm}		% nicer bold math symbols with \boldsymbol
% \usepackage{siunitx}		%units
% ..................................................
%% colors
% \usepackage{color}		%to set/define colors,
% predefined: white , black, red, green, blue, cyan, magenta, yellow
% ..................................................
%% objects
\usepackage{wrapfig,graphicx}	%inclusion of graphics with \includegraphics{name}
\usepackage{tikz,pgf}
\usetikzlibrary{arrows}
\usetikzlibrary{matrix}
\usetikzlibrary{decorations.pathmorphing}
\usepackage{framed}
% ..................................................
%% index/bibliography
\usepackage{makeidx}		%create an index with \makeindex in head
% (creates *.idx file)
\makeindex			%if index is required (after {makeidx})
\usepackage[backend=biber,style=alphabetic]{biblatex}	% print bibliography with \printbibliography
\bibliography{numerik_script.bib} % loads .bib-file for
% biber-bibliography; after {biblatex}
% ..................................................
%% (nicer)tables
\usepackage{booktabs} 	%enables better spacing and lines in tables
% \usepackage{multirow}	%option multirow for tables (centers text vertically)
% \usepackage{tabulary}	%table-width matches content
% ..................................................
%% font
\usepackage[normalem]{ulem}         % strike text out
% ..................................................
%% hyperlinks
\usepackage{hyperref}
% options: colorlink=true (-> no boxes but colored hyperlinks), color of all link
% ..................................................
%% fonts
% \usepackage{anyfontsize}	%any fontsize with \fontsize{size}{baselineskip}\selectfont
\usepackage{tabularx}  %table with extendable X-column

% DEFINITIONS
% ---------------------------------------------------------------
% math (theorems)
\theoremstyle{definition}
\newtheorem{Def}{Definition}[section]		%definition with \begin{Def}
\newtheorem{Bsp}[Def]{Beispiel}
% \theoremstyle{remark}
\newtheorem{Bem}[Def]{Bemerkung}	%corollary with \begin{Korr}
\newtheorem{Wdh}[Def]{Wiederholung}
\theoremstyle{plain}
\newtheorem{Satz}[Def]{Satz}		%theorem with \begin{Satz}
\newtheorem*{satz}{Satz}
\newtheorem{Lem}[Def]{Lemma}		%lemma with \begin{Lem}
\newtheorem{Kor}[Def]{Korollar}		%theorem with \begin{Satz}
\newtheorem{Fol}[Def]{Folgerung}
% \newtheorem{Ax}[Def]{Axiom}		%axiom with \begin{Ax}
% \newtheorem{Prop}[Def]{Proposition}%proposition with \begin{Prop}

% colors
% \definecolor{ashgrey}{rgb}{0.7, 0.75, 0.71}


% NEW COMMANDS
% ---------------------------------------------------------------
% symbols
\newcommand{\C}{\mathds{C}}
\newcommand{\R}{\mathds{R}}
\newcommand{\Ren}{\mathds{R}^{n}}
\newcommand{\Renn}{\mathds{R}^{n\times n}}
\newcommand{\Renm}{\mathds{R}^{n\times m}}
\newcommand{\Q}{\mathds{Q}}
\newcommand{\N}{\mathds{N}}
\newcommand{\Z}{\mathds{Z}}
\newcommand{\F}{\mathds{F}}
\newcommand{\p}{\mathcal{P}}
\newcommand{\K}{\mathcal{K}}

% for the three boxes to show float numbers
\newcommand{\floatbox}[3]{ %
  \begin{array}{|c|c|c|}
    \cline{1-3} 	
    #1 & #2 & #3\\
    \cline{1-3}
  \end{array}
}

\newcommand{\nn}[1]{\left\| #1 \right\|}	% norm
\newcommand{\scp}[2]{\langle #1, #2 \rangle}	% scalar prod. 

% inserted sections a), b)
\newcommand{\extrasection}[2]{\vspace{1.5eM}\minisec{\Large\itshape \thesection #1 #2}\vspace{1eM}}

% EVIL HACKS to go along with numeration
% equation numeration
\newcommand{\sectione}[1]{ \setcounter{equation}{0}\section{#1}}
% theorem numeration
\newcommand{\subsectione}[1]{\addtocounter{Def}{1}\subsection{#1}}
\newenvironment{Satze}[1][]{ %
  \begin{Satz}[#1]  }
  { \end{Satz}
  \addtocounter{subsection}{1}}
\newenvironment{Leme}[1][]{ %
  \begin{Lem}[#1] }
  {\end{Lem}
  \addtocounter{subsection}{1}}
\newenvironment{Kore}[1][]{ %
  \begin{Kor}[#1]}
  {\end{Kor}
  \addtocounter{subsection}{1}}
\newenvironment{Fole}[1][]{ %
  \begin{Fol}[#1]}
  {\end{Fol}
  \addtocounter{subsection}{1}}
\newenvironment{Bspe}[1][]{ %
  \begin{Bsp}[#1]}
  {\end{Bsp}
  \addtocounter{subsection}{1}}
\newenvironment{Beme}[1][]{ %
  \begin{Bem}[#1]}
  {\end{Bem}
  \addtocounter{subsection}{1}}
\newenvironment{Defe}[1][]{ %
  \begin{Def}[#1]}
  {\end{Def}
  \addtocounter{subsection}{1}}
\newenvironment{Wdhe}[1][]{ %
  \begin{Wdh}[#1]}
  {\end{Wdh}
  \addtocounter{subsection}{1}}

% Pseudocode
\newenvironment{pseudocode}[1]{ %
  \begin{minipage}{#1}
    \begin{framed}
      \hspace*{1em}	
      \begin{minipage}{#1}
        \begin{tabbing}
          for~~\= for~~\= for~~\= for~~\= some really really large amount of text \= \kill
	}
	{ %
        \end{tabbing}
      \end{minipage}
      \hspace*{1em}
    \end{framed}
  \end{minipage}
}

% IMAGES
\newcounter{imagecnt}[section]

% existing images
\newenvironment{image}[1]{%

\begin{minipage}{\linewidth}
\centering 
\textbf{Abb. \thesection.\arabic{imagecnt}}
\textit{~~#1}
\\[0.5eM]
}%
{
  \end{minipage}
  \stepcounter{imagecnt}\\[1eM]
}

% missing images
\newcommand{\imagemissing}[1]{
  \begin{image}{#1}
    \includegraphics[width=2cm]{images/image_missing.jpg}
  \end{image}
}	


% STYLE SETTINGS
% ---------------------------------------------------------------
% \cfoot[<text for scrplain>]{<text for scrheadings>}	
% \allowdisplaybreaks              %allow multipage for equations
\renewcommand{\theequation}{\thesection.\arabic{equation}}

%%% Local Variables:
%%% mode: latex
%%% TeX-master: "numerik_script"
%%% End:
