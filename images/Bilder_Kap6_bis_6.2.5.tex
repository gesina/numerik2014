\input{praeambel_gesina.tex}

\usepackage{pgf,tikz}
\usetikzlibrary{arrows}


\begin{document}
\centering

%Erste Grafik (Veranschaulichung stückweiser polynomialer Approximation) in 6.2 (Anfang)
	\begin{tikzpicture}[>=triangle 45]
		\draw[->,color=black] (0,0) -- (10,0);
		\draw[->,color=black] (0,0) -- (0,8);
		\clip(-0.5,-0.5) rectangle (10,8);
		
		\foreach \x/\text in {0.5/$a=x_0$,3.5/$x_1$,4.5/$x_2$,6/$x_3$,8.5/$x_4=b$}
			\draw (\x,0.1) -- (\x,-0.1) node[below] {\text};
		
		%k=1
		\foreach \x/\y/\wert in {0.5/3.5/4.5,3.5/4.5/1,4.5/6/1.5,6/8.5/5}
		{
			%Waagrechte Linien (chi-Funktionen)
				\draw(\x,\wert)--(\y,\wert);	
			%Beginn und Ende der Linien
			\draw(\x+0.1,\wert-0.2)--(\x,\wert-0.2)--(\x,\wert+0.2)--(\x+0.1,\wert+0.2);	
			\draw (\y-0.1,\wert-0.2) arc (-90:90:0.1 and .2);
		}
		
		%k=2 (linear)
		\foreach \x/\y/\wertx/\werty in {0.5/3.5/4.5/1,3.5/4.5/1/1.5,4.5/6/1.5/5,6/8.5/5/3.5}
			%Gerade zwischen den beiden Punkten
			\draw[color = blue] (\x,\wertx) -- (\y,\werty);
			
		\draw[color=red,smooth,samples=200,domain=0.5:6] plot(\x,{0.058*\x^3-0.073*\x^2-1.696*\x+5.359});
		\draw[color=red,smooth,samples=200,domain=6:8.55] plot(\x,{-0.293*\x^3+4.84*\x^2-24.067*\x+38.52});
	\end{tikzpicture}

%Die Grafik unter Formel (6.2.2) konnte ich im handschriftlichen Skript nicht finden
%Grafik zu Beispiel 6.2.4
	\begin{tikzpicture}[>=triangle 45]
			\draw[->,color=black] (0,0) -- (10,0);
			\draw[->,color=black] (0,0) -- (0,5);
			\clip(-0.5,-0.5) rectangle (10,5);
			
			\foreach \x/\text in {0.5/$x_0$,3.5/$x_1$,4.5/$x_2$,6/$x_3$,8.5/$x_4$}
				\draw (\x,0.1) -- (\x,-0.1) node[below] {\text};
			
			%k=1
			\foreach \x/\y/\wert in {0.5/3.5/3.5,3.5/4.5/1,4.5/6/1.5}
			{
				%Waagrechte Linien (chi-Funktionen)
					\draw(\x,\wert)--(\y,\wert);	
				%Beginn und Ende der Linien
				\draw(\x+0.1,\wert-0.2)--(\x,\wert-0.2)--(\x,\wert+0.2)--(\x+0.1,\wert+0.2);	
				\draw (\y-0.1,\wert-0.2) arc (-90:90:0.1 and .2);
			}
				
				\draw (6,4)--(8.5,4);
				\draw(6+0.1,4-0.2)--(6,4-0.2)--(6,4+0.2)--(6+0.1,4+0.2);
				\draw(8.5-0.1,4-0.2)--(8.5,4-0.2)--(8.5,4+0.2)--(8.5-0.1,4+0.2);
		\end{tikzpicture}

\end{document}