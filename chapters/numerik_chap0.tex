% % % % % % % % % % % % % % 
% 
% Skript zu NUMERIK I
% WS14/15
% von Prof. Dr. Blank
% Universität Regensburg
% 
% 
%	Kap. 0: Vorwort
% 
% % % % % % % % % % % % % % 


\chapter*{Vorwort}
% ---------------------------------------------------
\subsubsection{Skriptfehler}
An alle, die gedenken dieses Skript zur Numerikvorlesung im WS2014/15 zu 
nutzen:\\
Es wird keinerlei Anspruch auf Richtigkeit, Vollständigkeit und auch sicher nicht Schönheit
(ich bin \LaTeX-Anfänger) dieses Dokuments erhoben.

Ihr würdet mir aber unglaublich weiterhelfen, wenn ihr jede Anmerkung 
-- das kann alles, von groben inhaltlichen 
Fehlern über Rechtschreibkorrekturen bis hin zu Wünschen/Anregungen/Tipps zur Typografie, sein --
an mich weiterleitet!

Jegliche Anmerkungen bitte gleich und jederzeit an:
\begin{center}
  \textbf{\large
    gesina.schwalbe@stud.uni-regensburg.de}
\end{center}
\hspace{1cm}


\subsubsection{Bilder oder \enquote{IMAGE MISSING}}
Ich selber bin leider nur wenig mit (ordentlicher) Grafikerstellung in
\LaTeX vertraut,
aber dank Josef Wimmer stehen inzwischen die inhaltsrelevanten
Abbildungen zur Verfügung
-- an dieser Stelle nochmal herzlichen Dank!
Die noch fehlenden Grafiken sind mit \enquote{IMAGE MISSING} markiert und
wir sind dankbar über jede Mithilfe beim Füllen der Lücken:\\
Ihr könnt jederzeit \textbf{die entsprechenden Bilder an mich schicken}!\\
Dann werden sie an die entsprechenden Stellen eingebunden
bzw. digital abgezeichnet.
Und gerade bei den \textbf{Funktionsplots} (z.B. zum Newton-Verfahren)
würde ich mich sehr über Plot-Bilder freuen :-)


\subsubsection{Copyright}
Was das Rechtliche angeht bitte beachten: \\
Urheber des Inhalts dieser Mitschrift ist Prof. Dr. Luise Blank.
Dies ist nur eine genehmigte Vorlesungsmitschrift und unterliegt dem deutschen
Urheberrecht, jegliche nicht rein private Verwendung muss demnach vorher mit
Frau Blank abgesprochen werden.

Alle Grafiken, die mit \copyright ~gekennzeichnet sind, stehen unter
dem Urheberrecht von Josef Wimmer, alle Rechte vorbehalten.


\subsubsection*{Danksagung}
Vielen Dank an
\begin{description}
\item[Kerstin Blomenhofer] für die fleißige und ordentliche Mitschrift\\
  (und natürlich auch allen anderen, die mir Notizen zur Verfügung gestellt haben)
\item[Josef Wimmer] für die Unterstützung bei der Grafikerstellung
\item[Oliver Rümpelein] für den Großteil meiner \LaTeX-Kenntnisse und
  die tatkräftige Unterstützung bei allen Fragen zu allen Zeiten
\end{description}

%%% Local Variables:
%%% mode: latex
%%% TeX-master: "../numerik_script"
%%% End:
