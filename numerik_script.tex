\documentclass[ngerman,fontsize=11pt, paper=a4, parskip=false, titlepage=false, toc=bib]{scrbook}
%%options:
%	ngerman:	provides german headers etc.; default english
%	fontsize:	use 10-12pt
%	paper:		use a4
%	parskip: 	false sets to 1em
%	titlepage:	titlepage (true) or titlehead (false)
%	toc:		table of contents options
%
%
%ESSENTIAL PACKAGES
%-----------------------------------------------------------------------section-------------------------
%..................................................
%%encodings
\usepackage[T1]{fontenc}		%font-encoding WITH e.g. ö (default is OT1 with 7- instead of 8-bit)
\usepackage[utf8]{inputenc}		%input-encoding WITH e.g. ö (depends on system), better after fontenc
%
%..................................................
%%language
\usepackage{babel}				%rules typical for chosen language(s)
%
%..................................................
%%font-settings
\usepackage{lmodern}			%font: latin modern
\usepackage{microtype}			%micro-typographic optimizing, e.g. ligatures
%
%
%
%EXTRA PACKAGES
%------------------------------------------------------------------------------------------------\\
%..................................................
%%font-settings
\usepackage{scrpage2}			%KOMA pagestyles scrheadings, scrplain
%
%..................................................
%%quotes
\usepackage{csquotes}			%easy quotes with \enquote{...}
%
%..................................................
%%maths
\usepackage{amsmath}			%improves maths-sections
\usepackage{mathtools}                  %provides tools like \DeclareParedDelimiter or \DefineMathOperator
%\usepackage{amsthm}			%for proofs etc.
\usepackage{amssymb}			%further maths symbols like \square for proofs
\usepackage{dsfont}			%for \mathds{letter} to create e.g. the rational numbers symbol Q
\usepackage{stmaryrd}			%for math symbols like lightning bold
%\usepackage{mathrsfs}			%for calligraphic math symbols with \mathscr{letter}
%\usepackage{esint}				%for different sorts of integral signs like \landupint
%
%\usepackage{siunitx}			%units
%
%
%..................................................
%%pagestyle
%\usepackage{scrpage2}			%KOMA-script specialty for headings, footnotes, extends pagestyle
%\usepackage{enumitem}			%provides nicer items for enumeration, like \alpha*)
\usepackage{enumerate}
%
%..................................................
%%colors
%\usepackage{color}				%to set/define colors, predefined: white , black, red, green, blue, cyan, magenta, yellow
%
%..................................................
%%objects
%	\usepackage{graphicx}			%for inclusion of graphics with \includegraphics{name} command
%	\usepackage{placeins}			%to set barriers for floating objects with \FloatBarrier, to add to commands, list them in as options, e.g. \usepackage[section]{placeins}
%
%..................................................
%%index/bibliography
	\usepackage{makeidx}			%enables to create an index with \makeindex in head, creates *.idx file
		\makeindex					%if index is required (after {makeidx})
%	\usepackage[backend=biber]{biblatex}	%enables to print bibliography with \printbibliography, needs \bibliography{bib files}
%		\bibliography{bib}					%loads .bib-file for biber-bibliography after {biblatex}
%
%..................................................
%%(nicer)tables
%	\usepackage{booktabs} 			%enables better spacing and lines in tables
%	\usepackage{multirow}			%allows option multirow for tables (centers text vertically)
%	\usepackage{multicol}			%allows option multicol for tables
%	\usepackage{tabularx}  			%table with extendable X-column
%	\usepackage{tabulary}			%table-width matches content
%
%..................................................
%%(nicer) footnotes and marginpars
%	\usepackage[outer=4.5cm, marginparwidth=5cm]{geometry}		%provides commands to manipulate the dimension of several elements; needed for marginnote (outer is space width for margins)
%	\usepackage{marginnote}				%more individual marginpars with \marginnote{text}[width]; geometry package needed
%	\usepackage{todonotes}				%fancy styled (colored etc.) marginpars with \todo[options]{text}
%	\usepackage{footmisc}				%nicer footnotes
%
%..................................................

%strike text out
\usepackage{ulem}



%hyperlinks
	\usepackage{hyperref}			%for links/hyperlinks, better load last!, options: colorlink=true (-> no boxes but colored hyperlinks), color of all link
%
%
%..................................................
%fonts
%	\usepackage{anyfontsize}		%changes the fontsize with \fontsize{size}{baselineskip}\selectfont to any size





%DEFINITIONS
%------------------------------------------------------------------------------------------------
%..................................................
%math (theorems)
%\theoremstyle{definition}
%\newtheorem{Def}{Definition}		%definition with \begin{Def}
%\newtheorem{Ax}[Def]{Axiom}		%axiom with \begin{Ax}
%\newtheorem{Satz}[Def]{Satz}		%theorem with \begin{Satz}
%\newtheorem{Prop}[Def]{Proposition}%proposition with \begin{Prop}
%\newtheorem{Lem}[Def]{Lemma}		%lemma with \begin{Lem}
%\newtheorem{Korr}[Def]{Korrolar}	%corollary with \begin{Korr}

%colors
%\definecolor{ashgrey}{rgb}{0.7, 0.75, 0.71}




%NEW COMMAMDS
%------------------------------------------------------------------------------------------------
%\newcommand{\<new command>}{<what it shall do>}
\newcommand{\R}{\mathbb{R}}
\newcommand{\Q}{\mathbb{Q}}
\newcommand{\N}{\mathbb{N}}
\newcommand{\Z}{\mathbb{Z}}




%STYLE SETTINGS
%------------------------------------------------------------------------------------------------
%\pagestyle{scrplain}				%set pagestyle to scrheadings or scrplain (instead of headings or plain)
%\clearscrplain						%clear old style (either scrplain odr scrheadings)
%\cfoot[<text for scrplain>]{<text for scrheadings>}	%any new settings for foots/ headings
\allowdisplaybreaks              %allow multipage for equations
\renewcommand{\theequation}{\thesection.\arabic{equation}}




%************************************************************************************************
\begin{document}


%----------------------------------------------------------------------------------------------
%FRONTMATTER
%----------------------------------------------------------------------------------------------
%\frontmatter	%for book only, part for title etc.

%TITLE(PAGE)
%	\titlehead{titlehead (free)}
	\title{Skript Numerik I}
	\subtitle{bei Prof. Dr. Blank im WS14/15}
%	\subject{type}
	\author{Gesina Schwalbe}
%	\date{date}
%	\extratitle{Schmutztitel}
%	\publishers{Verlag}
%	\uppertitleback{Titelrückseitenkopf }
%	\lowertitleback{Titelrückseitenfuß}
%	\dedication{Widmung}
%	\thanks{Fußnote}
\maketitle

%---------------------------------------------------
\tableofcontents



%----------------------------------------------------------------------------------------------
%ACTUAL CONTENT
%----------------------------------------------------------------------------------------------
%\mainmatter		%for book only, part for main content of document
\chapter{Einführung}

\chapter{Lineare Gleichungssysteme: Direkte Methoden}
\label{2.1}
Sei $ A \in \R^{n\times n}$, $b \in \R^n$. Gesucht ist $x\in \R^n$ mit 
\begin{gather*}
	A\cdot x = b
\end{gather*}
Weitere Voraussetzungen sind die Existenz und Eindeutigkeit einer Lösung.
Bemerkung:
\begin{itemize}
	\item Ein verlässlicher Lösungsalgorithmus überprüft dies und behandelt alle Fälle. 
	\item Die Cramersche Regel ist ineffizient (s. Einführung).
	\item Das Inverse für $x=A^{-1}\cdot b$ aufzustellen ist ebenso ineffizient, denn es ist keine Lösung für alle $b\in \R^n$ verlangt und der Algorithmus wird evtl. instabil aufgrund vieler Operationen.
	\item [$\Rightarrow$] Invertieren von Matrizen vermeiden!!
	\item [$\Rightarrow$] Lösen des Linearen Gleichungssystems!!
\end{itemize}

\section{Gaußsches Eliminationsverfahren} \label{2.1} \index{Gaußsches Eliminationsverfahren}\index{Dreieckszerlegung}
Das Verfahren wurde 1809 von Friedrich Gauß, 1759 von Josepf Louis Lagrange beschrieben und war seit dem 1. Jhd. v. Chr. in China bekannt.

\subsection{Vorwärtselimination} \label{2.1.1}\index{Vorwärtselimination}\index{Vorwärtssubstitution}
Das Gaußverfahren gilt der Lösung eines linearen Gleichungssystems der Form
\begin{align*}
	Ax &= b
\end{align*}
mit $A=(a_{ij})_{i,j \leq n} \in K^{n\times n}$ Matrix und $b=(b_i)_{i\leq n} \in K^n$ Vektor.\\
Der zugehörige Algorithmus sieht folgendermaßen aus:
\begin{gather*}
	\begin{array}{ccccccccc}
	a_{11}x_1 &+& a_{12}x_2 &+& \cdots &+& a_{1n}x_n & = & b_1 \\
	a_{21}x_1 &+& a_{22}x_2 &+& \cdots &+& a_{2n}x_n & = & b_2 \\
	\vdots         &&        \vdots     &&              &&   \vdots       &    & \vdots \\
	a_{n1}x_1 &+& a_{n2}x_2 &+& \cdots &+& a_{nn}x_n & = & b_n \\\\
	&&&& \Arrowvert &&&& 
	\end{array} \\
\quad 	(\text{i-te Zeile}) - (\text{1. Zeile})\cdot \frac{a_{i1}}{a_11} \Rightarrow a_{i1}=0\\
\begin{array}{ccccccccc}
&&&& \Downarrow &&&&  \\\\
a_{11}x_1 &+& a_{12}x_2 &+& \cdots &+& a_{1n}x_n & = & b_1 \\
				  &+& a_{22}^{(1)}x_2 &+& \cdots &+& a_{2n}^{(1)}x_n & = & b_2^{(1)} \\
				 &&        \vdots     &&              &&   \vdots       &    & \vdots \\
														&& && && a_{nn}^{(1)}x_n & = & b_n^{(1)} \\\\
&&&& \Downarrow &&&&\\
&&&& \vdots &&&&
\end{array} 
\end{gather*}
mit
\begin{align*}
	a_{ij}^{(1)} &= a_{ij}-a_{1j}\cdot \frac{a_{i1}}{a_{11}} & \text{für }i,j = 2, \cdots, n \\
	b_i^{(1)}      &= b_i- b_1\cdot \frac{a_{i1}}{a_{11}}        & \text{für }i = 2, \cdots, n 
\end{align*}
In jedem Schritt werden die Einträge der $k$-ten Spalte analog unterhalb der Diagonalen (also $k=1, \cdots, n-1$) eliminiert:
\begin{align*}
	(\text{$i$-te Zeile})- (\text{$k$-te Zeile})\cdot\frac{a_{ik}}{a_{kk}} && \text{für } i=k+1, \cdots ,n 
\end{align*}
Die Reihe 
\begin{gather*}
			A \rightarrow A^{(1)} \rightarrow A^{(2)} \rightarrow \dotsm \rightarrow A^{(n-1)}
\end{gather*}
wird bis zum $n$-ten Schritt fortgeführt, d.h. bis eine obere Dreiecksgestalt eintritt:
\begin{align}
\nonumber
\underbrace{	\begin{pmatrix}
	a_{11} & \dotsm & \dotsm & a_{1n} \\
	             & a_{22}^{(1)} & \dotsm & a_{2n}^{(1)} \\
	             &&              \ddots  &  \vdots \\
	   0        && &                             a_{nn}^{(n-1)}
	\end{pmatrix}}_{\coloneqq R}
	\cdot
	\begin{pmatrix}
		x_1 \\
		x_2 \\
		\vdots \\
		x_n
	\end{pmatrix}
	& =
	\underbrace{\begin{pmatrix}
		b_1 \\
		b_2^{(1)} \\
		\vdots \\
		b_n^{(n-1)}
	\end{pmatrix}}_{\coloneqq z} \\
Rx &= z 	\label{II.1.1} 
\end{align}
wobei für  $i=k+1, \cdots ,n$ die Einträge wie folgt aussehen:
\begin{align}	
	l_{ik} &\coloneqq \frac{a_{ik}^{(k-1)}}{a_{kk}^{(k-1)}} \label{II.1.2} \\
	a_{ij}^{(k)} &= a_{ij}^{(k-1)} - a_{kj}^{(k-1)}\cdot l_{ik} \label{II.1.3}
				 & \text{für } j=k+1, \cdots , n\\ %
	b_i^{(k)} &= b_i^{(k-1)} -b_k^{(k-1)} \cdot   l_{ik}\label{II.1.4}\index{Vorwärtssubstitution}
\end{align}
Dieser Prozess wird \textbf{Vorwärtselimination} genannt.

\subsection{Rückwärtselimination}\label{2.1.2}
Für die Lösung des Gleichungssystems ist dann noch die \textbf{Rückwärtssubstitution} \index{Rückwärtssubstitution} nötig:
\begin{align}
	x_1 &= \frac{b_1^{(n-1)}}{a_{nn}^{(n-1)}} \label{II.1.5} \\
	x_{n-1} &=  \frac{b_{n-1}^{(n-2)}-a_{n-1,n}^{(n-1)}\cdot x_n}{a_{(n-1)(n-1)}^{(n-2)}} \label{II.1.6} \\
	x_k &= \frac{b_k^{(k-1)}-\sum_{j=k+1}^{n}a_{kj}^{(k-1)}x_j}{a_{kk}^{(k-1)}} \label{II.1.7}
\end{align}

\subsection{Vorsicht}
	Algorithmen \ref{2.1.1} und \ref{2.1.2} sind nur ausführbar, falls für die sog. \textbf{Pivotelemente $\mathbf{a_{kk}^{(k-1)}}$ } \index{Pivotelement} gilt:
	\begin{gather*}
			a_{kk}^{(k-1)} \neq 0 \quad   \text{für } k=1, \cdots , n
	\end{gather*}
	Dies ist auch für invertierbare Matrizen nicht immer gewährleistet.
	
\subsection{Weitere algorithmische Anmerkungen}	\label{2.1.4}
Matrix $A$ und Vektor $b$ sollten möglichst \textbf{nie} überschrieben werden! (Stattdessen kann eine Kopie überschrieben werden.) \\
Das Aufstellen von $A$ und $b$ ist bei manchen Anwendungen das teuerste, sie gehen sonst verloren. In \ref{2.1.1} wird das obere Dreieck von $A$ überschrieben. Dies ist möglich, da in \eqref{II.1.3} nur die Zeilen $k+1, \cdots, n$ mithilfe der $k$-ten bearbeitet werden. Am Ende steht $R$ im oberen Dreieck von $A$ und $z$ in $b$. \\
Die $l_{ik}$ werden spaltenweise berechnet und können daher anstelle der entsprechenden Nullen (in der Kopie) von $A$ gespeichert werden, d.h.:
\begin{gather}
	\widetilde{L} \coloneqq (l_{ik})  \label{II.1.8}
\end{gather}
und $R$ werden sukzessive in A geschrieben.
TIKZ MISSING
Der Vektor $z$ und anschließend der Lösungsvektor $x$ kann in (eine Kopie von) $b$ geschrieben werden.
Wird eine neue rechte Seite $b$ betrachtet, muss \ref{II.1.1} nicht komplett neu ausgeführt werden, da sich $\widetilde{L}$ nicht ändert. Es reicht \ref{II.1.4} zu wiederholen.
TIKZ MISSING

\subsection{Dreieckszerlegung} \label{2.1.5} \index{Dreieckszerlegung}
Die Dreieckszerlegung einer Matrix $A$ entspricht dem Verfahren aus \ref{2.1.1}, nur ohne die Zeile \eqref{II.1.4}.

\subsection{Vorwärtssubstitution} \index{Vorwärtssubstitution}
Die Vorwärtssubstitution entspricht der in \ref{2.1.4} bzw. dem Verfahren aus \ref{2.1.1} ohne die Bestimmung von $l_{ik}$ und $R$, also nur Schritt \ref{II.1.4}.

\subsection{Gauß-Eleminator zur Lösung von $Ax=b$}\index{Gauß-Eleminator}
\begin{enumerate}[1]
	\item Dreieckszerlegung
	\item Vorwärtssubstitution        $\quad b_i^{(k)} = b_i^{(k-1)} -b_k^{(k-1)} \cdot   l_{ik} $
	\item Rückwärtssubstitution      $\quad x_k = \frac{b_k^{(k-1)}-\sum_{j=k+1}^{n}a_{kj}^{(k-1)}x_j}{a_{kk}^{(k-1)}}$
\end{enumerate}

\subsection{Rechenaufwand gezählt in \enquote{flops}} \index{flops}\index{floating point operations}\index{Rechenaufwand}
\textbf{\enquote{flops} }= \textbf{f}loating \textbf{p}oint \textbf{op}eration\textbf{s} \\
MISSING

\subsection{Definition: Landau-Symbole} \index{Landau-Symbole}
MISSING

\subsection{Allgemeines zur Aufwandsbetrachtung}\index{Rechenaufwand}
MISSING

\subsection{Formalisieren des Gauß-Algorithmus} \index{Gaußsches Eliminationsverfahren} \index{LR-Zerlegung}\index{LU-Zerlegung}
MISSING

\subsection{Lemma (Eigenschaften der $L_k$-Matrizen)} \label{2.1.12} \index{Frobeniusmatrix}
MISSING


\subsection{Satz (LR- oder LU-Zerlegung)} \index{LR-Zerlegung}\index{LU-Zerlegung}
MISSING
\index{Verfahren von Crout}

\section{Gaußsches Eliminationsverfahren mit Pivotisierung}
MISSING
\subsection{Spaltenpivotisierung (=partielle/ halbmaximale Pivotisierung)} \index{Spaltenpivotisierung}\index{Pivotisierung}\index{halbmaximale, partielle Pivotisierung}

\subsection{Bemerkungen}
\index{Zeilenpivotisierung}
\index{Vollständig Pivotsuche}\index{vollständige Pivotisierung}
\index{Permutationsmatrix}

\subsection{Satz: Dreieckszerlegung mit Permutationsmatrix} \label{2.2.4} 

\paragraph{Beweis}~	\marginpar[15.10.2014 \\(Fortsetzung)]{15.10.2014 \\(Fortsetzung)}
	\begin{align*}
		PA &= LR  \\
		R &= A^{(n-1)}\\& = L_{n-1}P_{\tau_{n-1}}\dots L_1P_{\tau_1}A
		\end{align*}
	Da $\tau_i$ nur zwei Zahlen $\geq i $ vertauscht, ist
	\begin{align*}\index{Vorwärtselimination}
		\Pi_i  &\coloneqq \tau_{n-1} \circ \dots \tau_i \quad\text{ für } i=1,\dots (n-1) 
		\end{align*}
	eine Permutation der Zahlen $\{i,\dots, n\}$, d.h. insbesondere gilt:
	\begin{alignat*}{2}
		\Pi_i(j)&=j  & \quad &\text{ für } j=1,\dots,(i-1) \\
%	\intertext{sowie}
		\Pi_i(j)&\in \{i, \dots, n\} & &\text{~für~}j=i,\dots, n\,. \\
		P _{\Pi_{i+1}}  &= (e_1, \dotsc e_i, e_{\Pi_{i+i}(i+1)}, \dotsc, e_{\Pi_{i+1}(n)}) && \\
%								&= \left(\begin{array}{c@{\vdots \,\,}c}
%									I_i & 0 \\
%							%		\dotsm & \dotsm \\
%									0  & P_{\sigma}
%									\end{array}\right)
								&= \begin{pmatrix}
										I_i & 0 \\
										0 & P_{\sigma}
								\end{pmatrix}
	\end{alignat*}
	Damit folgt:
	\begin{align*}
	P_{\Pi_(i+1)}\cdot P_{\Pi_{i+1}}^{-1}  &= 
													P_{\Pi_{i+1}} \cdot \left(\begin{array}{ccc|ccc}
																						& I_i & && 0 & \\
																						\cline{1-6}
																						&     & -l_{i+1, i} & & & \\
																						&  0 &  \vdots      & & I_{n-i} &\\
																						&     & -l_{n, i} & &  & 
																					\end{array}\right)
																			\cdot \begin{pmatrix}
																					I_i & 0 \\
																					0 & P_{\sigma}^{-1}
																			\end{pmatrix}\\
	 &= \begin{pmatrix}
				 I_i & 0 \\
				 0 & P_{\sigma}
			 \end{pmatrix} \cdot   \l\cdot  \cdot  \left(\begin{array}{ccc|ccc}
													 & I_i & && 0 & \\
													 \cline{1-6}
												\cdot  	 &     & -l_{i+1, i} & & & \\
													 &  0 &  \vdots      & & P_{\sigma}^{-1} &\\
													 &     & -l_{n, i} & &  & 
												\end{array}\right) \\
	 &= \left(\begin{array}{ccc|ccc}
				 & I_i & && 0 & \\
				 \cline{1-6}
				 &     & -l_{\Pi{i+1}(i+1), i} & & & \\
				 &  0 &  \vdots      & & I_{n-i} &\\
				 &     & -l_{\Pi{i+1}(n), i} & &  & 
		 \end{array}\right) \\
	 &= I - (P_{\Pi_{i+1}} l_i)e_i^T\\
	 &\eqqcolon \widehat{L}_i
	\end{align*}
	und
	\begin{align*}		R =&\, L_{n-1}\\
					&\cdot (P_{\tau_{n-1}}L_{n-2}P_{\tau_{n-1}}^{-1})\\
		&				\cdot (P_{\tau_{n-1}}P_{\tau_{n-2}}L_{n-2}P_{\tau_{n-2}}^{-1}P_{\tau_{n-1}}^{-1})\\
		&\; \vdots \\
		&		 \cdot (P_{\tau_{n-1}}\dotsm P_{\tau_{1}}L_{1}P_{\tau_{1}}\dotsm P_{\tau_{n-1}}) \cdot A\\
=&\,L_{n-1}\widehat{L}_{n-2}\dotsm\widehat{L}_1P_{\Pi_{1}}\cdot A
\end{align*}
Nach Lemma \ref{2.1.12} gilt daher, es existiert eine Permutation $\Pi_{1}$ mit
\begin{gather*}
	P_{\Pi_1}\cdot A = LR ,
	\end{gather*}
wobei $R$ obere Dreiecksgestalt hat und
\begin{align*}
		L  &=  \begin{pmatrix}
						1 && & 0\\
						l_{\Pi_2(2),1} & \ddots & \\
						\vdots &            \ddots &  1\\
						l_{\Pi_n(n),1}& \dotsm &  l_{\Pi_n(n),n-1} & 1 
					\end{pmatrix} 
					& \text{mit } |l_{ij}| \leq 1 
\end{align*}
gilt.

\subsection{Lösen eines Gleichungssystems $Ax=b$} \label{2.2.5}
\subsection{Bemerkungen}
\subsection{Beispiel zur Pivotisierung}



\chapter{Fehleranalyse} \index{Fehler}
MISSING
\index{Stabilität des Algorithmus}
\section{Zahlendarstellung und Rundungsfehler} \label{3.1} \index{Fehler} \index{Rundungsfehler}\index{Zahlendarstellung}
\subsection{Definition: Gleitkommazahl} \label{3.1.1} \index{Gleitkommazahl}
Eine Zahl $x\in\Q$ mit einer Darstellung
\begin{align*}
	x&=\sigma \cdot(a_1 . a_2 \cdots a_t)_{\beta}\cdot \beta^e 
	 = \sigma\beta^e \cdot \sum_{\nu=1}^{t}a_\nu \beta^{-\nu+1}\\\\
&\quad\begin{array}{ll}
	\beta\in\N & \text{Basis des Zahlensystems}\index{Basis}\\
	\sigma \in\{\pm 1\} &\text{Vorzeichen} \\
	m = (a_1 . a_2 \cdots a_t)_{\beta} &\text{Mantisse}\index{Mantisse}\\
	\phantom{m}= \sum_{\nu=1}^{t}a_\nu \beta^{-\nu+1} \\
	a_i \in\{0,\cdots , \beta-1\}&\text{Ziffern der Mantisse}\\
	t\in\N&\text{Mantissenlänge} \\
	e\in\Z &\text{mit }e_{min}\leq e \leq e_{max} \text{ Exponent}
\end{array}
\end{align*}
heißt \textbf{Gleitkommazahl} mit $t$ Stellen und Exponent $e$ zur Basis $b$. \\
Ist $a_1\neq 0$, so heißt $x$ \textbf{normalisierte Gleitkommazahl}\index{normalisierte Gleitkommazahl}

\subsection{Bemerkung} \label{3.1.2}
\begin{enumerate}[a)]
	\item 0 ist keine normalisierte Gleitkommazahl, da $a_1 =  0$ ist.
	\item $a_1\neq 0$ stellt sicher, dass die Gleitkommadarstellung eindeutig ist.
	\item In der Praxis werden auch nicht-normalisierte Darstellungen verwendet.
	\item Heutige Rechner verwenden meist $\beta =2$, aber auch $\beta=8, \beta=16$.
\end{enumerate}

\subsection{Beispiel} \label{3.1.3}
 bit-Darstellung nach IEEE-Standard 754 von floating point numbers
 MISSING
 
\begin{align*}
\intertext{Betragsmäßig \textbf{größte Zahl}:}
	 \begin{array}{|c|c|c|}
	 \cline{1-3} 	
	 0 & 01\cdots 1& 1\cdots\cdots 1\\
	 \cline{1-3}
	 \end{array} && 
	 x_{max} = (2-2^{-23})\cdot 2^{127}  & \approx 3,4 \cdot 10^{38}
\intertext{Betragsmäßig \textbf{kleinste Zahl}:}
	\begin{array}{|c|c|c|}
	\cline{1-3} 	
	0 & 0\cdots 0& 0\cdots\cdots 01\\
	\cline{1-3}
	\end{array} && 
	x_{min} = (2-2^{-23})\cdot 2^{-126} = 2^{-149}  & \approx 1,4 \cdot 10^{-45}
\end{align*}

\subsection{Verteilung der Maschinenzahlen} \label{3.1.4}
ungleichmäßig im Dezimalsystem, z. B.
\begin{align*}
		x &= \pm a_1 . a_2 a_3 \cdot 2^e  & -2\leq & e\leq 1 & a_i & \in \{0,1\}  \\
		TIKZ MISSING
\end{align*}
ist im Dualsystem gleichmäßig verteilt.

\subsection{Bezeichnungen} \label{3.1.5}
\begin{itemize}
	\item[\textbf{overflow}] es ergibt sich eine Zahl, die betragsmäßig größer ist als die größte maschinendarstellbare Zahl
	\item[\textbf{underflow}] entsprechend, betragsmäßig kleiner als die kleinste positive Zahl
\end{itemize}
Bsp.: overflow beim integer $b=e+127$
\begin{align*}
	\begin{array}{rrr@{}}
	b &= 254                                &11111110 \\
	   &+  \phantom{24}3 &00000011 \\
	   \cline{3-3} %
	 b+3 = 257 \text{ mod } 2^8  &=\phantom{24}1& \xout{1}00000001 
	\end{array}	  
\end{align*}


\subsection{Rundungsfehler} \label{3.1.6}
Habe $x\in \R $ die normalisierte Darstellung
\begin{align*}
	x &= \sigma \cdot \beta^e (\sum_{\nu=1}^{t}a_{\nu}\beta{1-\nu} + \sum_{\nu=t+1}^{\infty}a_{\nu}\beta^{1-\nu} ) \\
	  &= \sigma \cdot \beta^e (\sum_{\nu=1}^{t}a_{\nu}\beta^{1-\nu} + \beta^{1-t}\sum_{l=1}^{\infty}a_{t+l}\beta^{l} )
\end{align*}
mit $e_{min} \leq e \leq e_{max}$, dann wird mit $fl(x)$ die gerundete Zahl bezeichnet, wobei $fl(x)$ 
eindeutig gegeben ist durch die Schranke an den \textbf{absoluten Rundungsfehler} \index{Fehler!absoluter Rundungsfehler}
\begin{align*}
	| fl(x) - x | \leq \begin{cases}
								\frac{1}{2}\beta^{e+1+t} & \text{bei symmetrischem Runden}\\
								\beta^{e+1+t}                    & \text{bei Abschneiden}
							\end{cases} \quad .
\end{align*}
Für die \textbf{relative Rechengenauigkeit} \index{Genauigkeit!relative Rechengenauigkeit}folgt somit 
\begin{align*}
\frac{| fl(x) - x | }{|x|} & \leq \begin{cases}
												\frac{1}{2}\beta^{1-t} & \text{bei symmetrischem Runden}\\
												\beta^{1-t}                    & \text{bei Abschneiden}
											\end{cases} \quad .
\end{align*}
Die \textbf{Maschinengenauigkeit} \index{Genauigkeit!Maschinengenauigkeit} des Rechners ist daher durch 
\begin{align*}
	  eps &= \beta^{1-t} & \text{(für float}\approx 10^{-7}  \text{, für double} \approx10^{-16} )
\end{align*}
gegeben.

Die Mantissenlänge bestimmt also die Maschinengenauigkeit. Bei einfacher Genauigkeit ist $fl(x)$ bis auf ungefähr 7 signifikante Stellen genau. \\
Im Folgenden betrachten wir symmetrisches Runden und definieren daher
\[ \tau \coloneqq \frac{1}{2}eps\]
Weiterhin gilt:
\begin{enumerate}[a)]
	\item Die kleinste Zahl am Rechner, welche größer als 1 ist, ist
					\[ 1 + eps \]
	\item Eine Maschinenzahl x repräsentiert eine Eingabemenge
					\[  E(x) = \{\widehat{x} \in \R : |\widehat{x}-x| \leq \tau|x|\} \]
					TIKZ MISSING
\end{enumerate}

\subsection{Bemerkung} \label{3.1.7}
Gesetze der arithmetischen Operationen gelten i.A. nicht, z.B.
\begin{itemize}
	\item 	$x$ Maschinenzahl $\quad \Rightarrow fl(x+\nu) = x \text{     für }|\nu| < \tau |x|$
	\item Assoziativ- und Distributivgesetze gelten nicht, z.B. für $\beta = 10, \, t=3, \, a=0,1 ,\, b= 105 , \, c= -104$ gilt:
					\begin{align*}
						fl(a+fl(a+c)) &= 1,1 \\
						fl(fl(a+b)+c) &= fl(fl(105,1) + (-104) ) \\
							                &= fl(105-104) \\
							                &= 1 \quad \lightning
					\end{align*}
	\item[ $\Rightarrow$] Für einen Algorithmus ist die Reihenfolge der Operationen wesentlich!
									  Mathematisch äquivalente Formulierungen können zu verschiedenen Ergebnissen führen.
\end{itemize}

\subsection{Auslöschung von signifikanten Stellen} \label{3.1.8}
Sei $x=9,995\cdot 10^{-1}, y=9,984 \cdot 10^{-1}$. Runde auf drei signifikante Stellen und berechne $x-y$:
\begin{align*}
	\widehat{f}(x,y) &\coloneqq fl(fl(x)- fl(y)) = fl(1,00\cdot 10^0 - 9,98\cdot 10^{-1}) \\
							  &= 	fl(0,02\cdot 10^{-1}) \\
							  &= fl(2,00 \cdot 10^{-3}) \\
	f(x,y)  &\coloneqq x-y \\
		      &\coloneqq 0,0011 = 1,1\cdot 10^{-3}
	\intertext{Daraus ergibt sich der relative Fehler}
	\frac{|\widehat{f}(x,y)-f(x,y)|}{|f(x,y)|}
		     &= \frac{|2\cdot 10^{-3}- 1,1\cdot 10^{-3}|}{|1,1\cdot 10^{-3}}
		       = 82\%
\end{align*}
Der Grund hierfür ist, dass das Problem der Substraktion zweier annähernd gleich großer Zahlen
schlecht konditioniert ist.

\textbf{Zwei Regeln:}
\begin{enumerate}[1)]
	\item Umgehbare Substraktion annähernd gleich großer Zahlen vermeiden!
	\item Unumgängliche Substraktion möglihcst an den Anfang des Algorithmus stellen! (s. später)
\end{enumerate}

% 2.2
%-----------------------------------------------------------------------------------------------------------------------------------------
\section{Kondition eines Problems} \label{3.2}
Es wird das Verhältnis 
\begin{gather*}
	\frac{\text{Ausgabefehler}}{\text{Eingabefehler}}
\end{gather*}
untersucht.

\subsection{Definition: Problem} \label{3.2.1} \index{Problem}
Sei $f: U \subseteq \R^n \mapsto \R^m$ mit $U$ offen und sei $x\in U$. Dann bezeichne $(f, x)$ das Problem, zu einem gegebenen $x$ die Lösung $f(x)$ zu finden.

\subsection{Definition: absoluter und relativer Fehler} \label{3.2.2} \index{Fehler}
Sei $x\in\R^n$ und $\widehat{x} \in \R^n$ eine Näherung an $x$. Weiterhin sei $||\cdot||$ eine Norm auf $\R^n$.
\begin{itemize}
	\item[a)] $||\widehat{x} - x||$ heißt \textbf{absoluter Fehler} \index{Fehler!absoluter}
	\item[b)] $\frac{||\widehat{x} - x||}{||x||}$ heißt \textbf{relativer Fehler}\index{Fehler!relativer}
\end{itemize}
Da der relative Fehler skalierungsinvariant ist, d.h. unabhänging von der  Wahl von $x$ ist, ist dieser i.d.R. von größerem Interesse.
Beide Fehler hängen von der Wahl der Norm ab!
Häufig werden Fehler auch komponentenweise gemessen:
\begin{align*}
	\text{Für } i=1,\cdots , n : && |\widehat{x}_i - x_i | & \leq \delta & \text{ (absolut)} \\
											 && |\widehat{x}_i - x_i | &\leq \delta |x_i| & \text{ (relativ)}
\end{align*}

\subsection{Wiederholung: Normen} \label{3.2.3}\index{Norm}

\begin{align*}
	\text{Euklidische Norm ($l_2$-Norm):} &&	||x||_2 &\coloneqq \sqrt{\sum_{i=1}^{n}|x_i|^2}
		\index{Norm!Euklidische Norm}
			TIKZ MISSING \\
	\text{Maximumsnorm ($l_\infty$-Norm):} &&	||x||_\infty &\coloneqq \max\{|x_i| : i=1, \cdots n\}
	  	\index{Norm!Maximumsnorm}
		  	TIKZ MISSING \\
	\text{Summennorm ($l_1$-Norm):} &&	||x||_1 &\coloneqq \sum_{i=1}^{n}|x_i| 
		\index{Norm!Summennorm}
			TIKZ MISSING \\
	\text{Hölder-Norm ($l_p$-Norm):} &&	||x||_p &\coloneqq 
		\left(\sum_{i=1}^{n}|x_i|^p\right)^{\frac{1}{p}} 
		\index{Norm!Hölder-Norm}
\end{align*}

\subsection{Definition: Matrixnorm} \label{3.2.4}
Auf dem $\R^n$  sei die Norm $||\cdot||_a$ und auf dem $\R^m$ die Norm $||\cdot||_b$ gegeben.
Dann ist die zugehörige \textbf{Matrixnorm} \index{Norm!Matrixnorm} gegeben durch:
\begin{align}
	||A||_{a,b} &\coloneqq \sup_{x\neq 0}- \frac{||Ax||_b}{||x||_a} \\ \nonumber
					 &= \sup_{||x||_a=1} ||Ax||_b \label{Matrixnorm} 
\end{align}
Also ist   $||A||_{a,b}||$ die kleinste Zahl $c>0$ mit
\begin{gather*}
	||Ax||_b  \leq c||x||_a \quad\quad \forall x\in \R^n
\end{gather*}

\subsection{Definition: Frobeniusnorm, p-Norm, Verträglichkeit} \label{3.2.5}
Sei $A\in \R^{m\times m}$.
\begin{enumerate}[a)]
	\item \textbf{Frobeniusnorm} (Schurnorm):
			 $ \quad ||A||_F \coloneqq \sqrt{\sum_{i=1}^{m}\sum_{j=1}^{n}|a_{ij}^2}
				 \index{Norm!Frobeniusnorm}$
    \item \textbf{p-Norm}: 
			 $\quad ||A||_p \coloneqq ||A||_{p,p}
				 \index{p-Norm}$
    \item Eine Matrixnorm heißt \textbf{verträglich} \index{Norm!verträglich} mit den Vektornormen 
			    $||\cdot||_a, ||\cdot||_b$, falls gilt
			    \footnote{ Beachte: $||A||_{a,b}$ ist die kleinste Norm im Gegensatz zu $||A||$, welche hier beliebig ist.}:
				 \begin{gather*}
					 	||Ax||_b \leq ||A|| \cdot ||x||_a \quad \forall x\in \R^n
				 \end{gather*}
\end{enumerate}

\subsection{Bemerkungen} \label{3.2.6}
\begin{enumerate}[a)]
	\item Die Normen $||\cdot||_F$ und $||\cdot||_p$ sind \textbf{submultiplikativ} \index{Norm!submultiplikativ}, d.h.
				\begin{gather*}
					||A\cdot B|| \leq ||A|| \cdot ||B||
				\end{gather*}
	\item Die \textbf{Spaltensummennorm}\index{Norm!Spaltensummennorm} ist definiert als
				\begin{gather*}
					||A||_1 = \max_{1\leq j \leq n}\sum_{i=1}^{m}|a_{ij}|
				\end{gather*}
			Sie ist das Maximum der Spaltensummen.
	\item Die \textbf{Zeilensummennorm} \index{Zeilensummennorm}ist äquivalent definiert als
				\begin{gather*}
					||A||_\infty = \max_{1\leq i \leq m}\sum_{j=1}^{n}|a_{ij}|
				\end{gather*}
	\item Die Frobeniusnorm $||\cdot||_F$ ist verträglich mit der euklidischen Norm $||\cdot||_2$
	\item Die Wurzeln aus den Eigenwerten von $A^TA$ heißen \textbf{Singulärwerte $\sigma_i$} \index{Singulärwert} von A. Sie definieren die Norm
				\begin{align*}
					||A||_2 &\coloneqq max \{\sqrt{\mu} : A^TA\cdot x = \mu x\text{ für ein }x\neq 0 \} \\
								& = \sigma_{max}
				\end{align*}
\end{enumerate}

%ultimate evil hack to go along with numeration
\minisec{\Large3.2 a) Normweise Konditionsanalyse} \label{3.2a}\vspace{1eM}

\subsection{Definition: absolute und relative Normweise Kondition}\index{normweise Kondition}
Sei $(f,x)$ ein Problem mit $f:U\in\R^n \rightarrow \R^m$
und $||\cdot||_a$ auf $\R^n$ und $||\cdot||_b$ auf $\R^m$ eine Norm.
\begin{enumerate}[a)]
	\item Die \textbf{absolute normweise Kondition}\index{Kondition!absolute normweise} eines Problems $(f,x)$ ist die kleinste Zahl 
			 $\kappa _{abs} > 0 $ mit
			 \begin{align}
			 	||f(\widehat{x})-f(x)||_b &\leq \kappa _{abs}(f,x) ||\widehat{x}-x||_a
			 	+ o\left(||\widetilde{x}-x||_a\right) \label{III.2.2} \\\nonumber
			 	\Bigl(f(\widetilde{x})- f(x) 
						 	&=\underbrace{ f'(x)(\widetilde{x}-x)+ o\left(||\widetilde{x}-x||\right)}_{Taylorentwicklung}
							 \quad \text{für }\widetilde{x}\rightarrow x 
						 	\Bigr)
			 \end{align}
	\item Die \textbf{relative normweise Kondition}\index{Kondition!absolute normweise} eines Problems $(f,x)$  mit $x\neq 0, f(x) \neq 0$
			 	ist die kleinste Zahl 
			 	$\kappa _{rel} > 0 $ mit
			 	\begin{align}
			 \frac{	||f(\widetilde{x})-f(x)||_b }{||f(x)||_b}
				 &\leq \kappa _{abs}(f,x)\frac{ ||\widetilde{x}-x||_a}{||x||_a}
			 	+ 
			 	\frac{||\widetilde{x}-x||_a}{||x||_a}) \label{III.2.3}
					 &&	\text{für } \widetilde{x} \rightarrow x
			 	\end{align}
	\item Sprechweise:
		\begin{itemize}\index{Kondition!gut/schlecht konditioniert}
			\item falls $\kappa$ \enquote{klein} ist, ist das Problem \enquote{gut konditioniert}
			\item falls $\kappa$ \enquote{groß} ist, ist das Problem \enquote{schlecht konditioniert}
		\end{itemize}
\end{enumerate}

\subsection{Lemma} \label{3.2.8}
Falls $f$ diffenrenzierbar ist, gilt
\begin{gather}
	\kappa_{abs}(f,x) = ||Df(x)||_{a,b} \label{III.2.4}
\end{gather}
und für $f(x) \neq 0$
\begin{gather}
	\kappa_{rel}(f,x) = \frac{||x_a||}{||f(x)||_b} -||Df(x)||_{a,b} \label{III.2.5}
\end{gather}
wobei $Df(x)$ die Jakobi-Matrix bezeichnet.
%
\subsection{Beispiel: Kondition der Addition} \index{Kondition der Addition}
$f(x_1, x_2) \coloneqq x_1 +x_2 , \, f:\R^2 \rightarrow \R$. \\
Wähle $l_1$-Norm auf $\R^2$ (und $\R$)
\begin{align*}
			Df(x_1, x_2) \, =(\triangledown f^T) \, &= (\frac{\partial}{\partial x_1}f, \frac{\partial}{\partial x_2}f )\\
				&= (1,1) && \text{(Matrix!)}
\end{align*}
damit
\begin{align*}
	\kappa_{abs} (f,x)&= ||Df(x)||_{1,1} && \text{(Matrix-Norm!!)}\\
						 		&= ||Df(x)||_1 \\
								&=1 \\
	\kappa_{rel} (f,x) &= \frac{||x||_1}{||f(x)||_1}| \cdot ||Df(x)||_{1} \\
								&= \frac{|x1| + |x_2|}{|x_1+x_2|}
\end{align*}
Daraus folgt: Die Addition zweier Zahlen mit gleichem Vorzeichen ergibt
\begin{gather*}
	\kappa_{rel} = 1
\end{gather*}
Die Subtraktion zweier annähernd gleich großer  Zahlen ergibt eine sehr schlechte relative
Konditionierung:
\begin{gather*}
\kappa_{rel} \gg 1
\end{gather*}
Zum Beispiel in \ref{3.1.8}: Es ist 
\begin{align*}
	x &= \begin{pmatrix}
		9,995 \\
		-9,984
	\end{pmatrix}
	\cdot 10^{-1} \\
	\widetilde{x} = fl(x) &= \begin{pmatrix}
		1 \\
		-9,98\cdot 10^{-1}
	\end{pmatrix}
\intertext{also}
	\frac{|f(\widetilde{x})-f(x)|}{|f(x)|}	&= \frac{0,9}{1,1} 
															= 0,\overline{81} \\
															&\leq \kappa_{rel}(f,x)\cdot \frac{||\widetilde{x}-x||_1}{||x||_1} \\
															&= \kappa_{rel}(f,x) \cdot 4,6\cdot 10^{-4}
\end{align*}
%

\subsection{Beispiel: Lösen eines Gleichungssystems}
Sei $A\in \R^{n\times n}$ invertierbar und $b\in \R^n$. Es soll 
\begin{gather*}
	Ax =b
\end{gather*}
gelöst werden.
Die mögliche Lösungen in $A$ und in $b$ lassen sich folgendermaßen ermitteln:
\begin{enumerate}[a)]
	\item Betrachte die Störungen in $b$:
			\begin{gather*}
			f: b\mapsto x= A^{-1}b 
			\end{gather*}
			Berechne dann $ \kappa(f,b)$ und löse 
			\begin{align*}
					A(x + \Delta x) &= b+\Delta b \\
					f(b + \Delta b) - f(b) &= \Delta x \\
													&= A^{-1} \cdot \Delta b && \text{da }x = A^{-1}b \\
					\Rightarrow ||\Delta x||  &= ||A^{-1}\Delta b|| \\
														&\leq ||A^{-1}||\cdot ||\Delta b|| && \forall b, \Delta b 
			\end{align*}
			wobei $||\cdot|| $ auf $\R^{n\times n}||$ die zum $||\R^n||$ zugeordnete Matrix-Norm sei. \\
			Die Abschätzung ist \textbf{scharf}, d.h. es gibt ein $\Delta b\in \R^n$, so dass \enquote{=} gilt, nach Definition \ref{3.2.4}. \\
			Also gilt
			\begin{gather}
				\kappa_{absl}(f,b) = ||A^{-1}|| \label{II.2.4}
			\end{gather}
			unabhängig von b $\left( x\mapsto Ax \quad \kappa_{abs}\right)$.
			Ebenso folgt die scharfe Abschätzung 
			\begin{align}
				\nonumber
				\frac{||	f(b + \Delta b) - f(b) - f(b)||}{||f(b)||} &= \frac{||\Delta x||}{||x||}\\ \nonumber
						& \leq  \frac{||A^{-1}||\cdot ||b||}{||x||} \cdot \frac{||\Delta b||}{||b||} \\
				\intertext{Damit}
				\kappa_{rel} (f,b) &= ||A^{-1} || \cdot \frac{||b||}{||A^{-1}\cdot b||} \label{III.2.7}
			\end{align}
			Da $||b|| \leq ||A||\cdot||x|| = ||A||\cdot||x|| = ||A||\cdot ||A^{-1}b||$ folgt:
			\begin{gather}
				\kappa_{rel}(f,b) \leq ||A|| \cdot ||A^{-1}|| \label{III.2.8}
			\end{gather}
			für alle (möglichen rechten Seiten) $b $.
			\ref{3.2.8} ist scharf in dem Sinne, dass es ein $\widehat{b}\in \R^n$ gibt 
			mit 
			\begin{gather*}
				||\widehat{b}|| = ||A||\cdot ||\widehat{x}||
			\end{gather*}
			und somit
			\begin{gather}
				\kappa_{rel}(f,\widehat{b}) = ||A||\cdot ||A^{-1}||
			\end{gather} %
			%
	\item betrachte Störungen in $A$:\\
			löse also 
			\begin{align*}
				(A+\Delta A)(x+\Delta x) &= b \\
				f:A&\mapsto x= A^{-1}b \\
				\R^{n\times n}&\rightarrow \R^n
			\end{align*}
			und berechne $\kappa(f,A)$ mittels Ableitung $Df(A):\R^{n\times n} \rightarrow \R^n$:
			\begin{align*}
				C\mapsto Df(A) C&= \frac{d}{dt} \left((A+tC)^{-1} \cdot b\right) \Big\vert_{t=0} \\
										  & = \frac{d}{dt}\left((A+tC)^{-1}\right)\Big\vert_{t=0}\cdot b
			\end{align*}			
			Weiterhin gilt
			\begin{align*}
				\frac{d}{dt} \left((A+tC)^{-1}\right) \Big\vert_{t=0} &= -A^{-1}CA^{-1} \label{III.2.9}
			\end{align*}
			da
			\begin{align*}
				0&= \frac{d}{dt}MISSING \\
				  &= \frac{d}{dt}\left( (A+tC)(A+tC)^{-1}\right)\\
				  &= C(A+tC)^{-1} +(A+tC)\cdot \frac{d}{dt}(A+tC)^{-1} \\
				  &\Rightarrow \frac{d}{dt} (A+ tC)^{-1} \\
				  &= -(A+tC)^{-1} \cdot C(A+tc)^{-1} \, ,
			\end{align*}
			falls $(A+tC)$ invertierbar ist. Für $t$ klein genug ist das gewährleistet, da $A$ invertierbar ist (s. Lemma \ref{3.2.12}) MISSING.
			\begin{gather*}
				\Rightarrow Df(A) C = -A^{-1}CA^{-1}b
			\end{gather*}
			Somit folgt
			\begin{align*}
				\nonumber
				\kappa_{abs} (f,A) &= ||Df(A)|| \\ \nonumber
											  &= \sup_{\substack{
																C\neq 0 \\ \nonumber
																C \in \R^{n\times n}														  	
														  	}}
														  \frac{||A^{-1}CA^{-1}b||}{||C||} \\ \nonumber
											  &= \sup_{\substack{
															 	C\neq 0 \\
															 	C \in \R^{n\times n}														  	
															 }}
												 \frac{||A^{-1}||\cdot||C||\cdot||A^{-1}b||}{||C||} \\
											  &= ||A^{-1}|| \cdot||x||
											  &\leq   ||A^{-1}||^2 \cdot||b||\\
				 \kappa_{rel}(f,A)  &= \frac{||A||}{||f(A)||} \cdot ||Df(A)|| \\\nonumber
											 &\leq ||A||\cdot ||A^{-1}|| \label{III.2.10}
			\end{align*}
	 \item betrachte Störungen in $A$ und $b$ :
		 \begin{gather*}
		 	(A+\Delta A)(x+\Delta x) = (b+\Delta b) 
		 \end{gather*}
		 Für $\kappa$ müsste $||(A,b)||$ festgelegt werden. Dies wird jedoch nichgt betrachtet. Es gilt jedoch folgende Abschätzung für invertierbare Matrizen $A\in \R{n\times n} $ und Störungen
		 $\Delta A \in \R^{n\times n}$ mit $||A^{-1}||\cdot ||\Delta A|| < 1$:
		 \begin{align*}
			 \frac{||\Delta x||}{||x||} & \leq ||A|| \cdot ||A{-1}||\cdot (1- ||A^{-1}\cdot ||\Delta A||) 
													 \cdot \underbrace{\neq  \frac{||(\Delta A, \Delta b)||}{||(A,b)||} }{\left(  \frac{||\Delta b||}{||b||} +  \frac{||\Delta A||}{||A||}  \right)}
													 \label{III.2.11}
		 \end{align*}
		 \textbf{Beweis:} s. Übungsblatt
\end{enumerate}

\subsection{Definition: Kondition einer Matrix} \index{MISSING}
Sei $||\cdot||$ eine Norm auf $\R{n\times n} $ und $A\in \R{n\times n}$ eine reguläre Matrix.
Die Größe
\begin{gather*}
	\mathbb{\kappa_{||\cdot||}(A) }= cond_{||\cdot||} \coloneqq ||A|| \cdot ||A^{-1}||
\end{gather*}
heißt \textbf{Kondition der Matrix} bzgl. der Norm ${||\cdot||}$. \\
Ist  ${||\cdot||}$ von einer Vektor-Norm ${||\cdot||}_p$ induziert, bezeichnet 
	$\mathbb{cond_p(A)}$
die $cond_{||\cdot||_p}(A)$. Wir schreiben $cond(A)$ für $cond_2(A)$. \\
$cond_{||\cdot||}(A) $ schätzt die relative Kondition eines linearen GLS $Ax=b$ für alle möglichen Störungen in $b$ oder in $A$ ab und diese Abschatzung ist scharf. \\
Es stellt sich nun die Frage: \\
\textit{Wann existiert die Inverse der gestörten invertierbaren Matrix $A$?}
\begin{align*}
	A+\Delta A &= A (I+A^{-1}\Delta A)\\
	||c|| < 1 \\
	(I-C)^{-1} &= \sum_{k=0}^{\infty}C^k \\
	||	(I-C)^{-1} || &\leq \frac{1}{1-||C||}
\end{align*}


%
%\begin{gather*}
%\int \dotsi \int \\
%+ \dotsb + \\
%, \dotsc , \\
%\dotso
%\end{gather*}

%----------------------------------------------------------------------------------------------
%BACKMATTER
%----------------------------------------------------------------------------------------------
%\backmatter		%for book only, part for index etc.


\printindex		%only with package makeidx
%\listoffigures		
%\listoftables
%\printbibliography	%only with package biblatex


\end{document}
%**********************************************************************************************









