\documentclass[ngerman,fontsize=11pt, paper=a4, parskip=false, titlepage=true, toc=bib]{scrbook}
%%options:
%	ngerman:	provides german headers etc.; default english
%	fontsize:	use 10-12pt
%	paper:		use a4
%	parskip: 	false sets to 1em
%	titlepage:	titlepage (true) or titlehead (false)
%	toc:		table of contents options
%
%
%ESSENTIAL PACKAGES
%------------------------------------------------------------------------------------------------
%..................................................
%%encodings
\usepackage[T1]{fontenc}		%font-encoding WITH e.g. ö (default is OT1 with 7- instead of 8-bit)
\usepackage[utf8]{inputenc}		%input-encoding WITH e.g. ö (depends on system), better after fontenc
%
%..................................................
%%language
\usepackage{babel}				%rules typical for chosen language(s)
%
%..................................................
%%font-settings
\usepackage{lmodern}			%font: latin modern
\usepackage{microtype}			%micro-typographic optimizing, e.g. ligatures
%
%
%
%EXTRA PACKAGES
%------------------------------------------------------------------------------------------------\\
%..................................................
%%font-settings
\usepackage{scrpage2}			%KOMA pagestyles scrheadings, scrplain
%
%..................................................
%%quotes
\usepackage{csquotes}			%easy quotes with \enquote{...}
%
%..................................................
%%maths
\usepackage{amsmath}			%improves maths-sections
\usepackage{mathtools}                  %provides tools like \DeclareParedDelimiter or \DefineMathOperator
%\usepackage{amsthm}			%for proofs etc.
\usepackage{amssymb}			%further maths symbols like \square for proofs
\usepackage{dsfont}			%for \mathds{letter} to create e.g. the rational numbers symbol Q
\usepackage{stmaryrd}			%for math symbols like lightning bold
%\usepackage{mathrsfs}			%for calligraphic math symbols with \mathscr{letter}
%\usepackage{esint}				%for different sorts of integral signs like \landupint
%
%\usepackage{siunitx}			%units
%
%
%..................................................
%%pagestyle
%\usepackage{scrpage2}			%KOMA-script specialty for headings, footnotes, extends pagestyle
%\usepackage{enumitem}			%provides nicer items for enumeration, like \alpha*)
\usepackage{enumerate}
%
%..................................................
%%colors
%\usepackage{color}				%to set/define colors, predefined: white , black, red, green, blue, cyan, magenta, yellow
%
%..................................................
%%objects
%	\usepackage{graphicx}			%for inclusion of graphics with \includegraphics{name} command
%	\usepackage{placeins}			%to set barriers for floating objects with \FloatBarrier, to add to commands, list them in as options, e.g. \usepackage[section]{placeins}
%
%..................................................
%%index/bibliography
	\usepackage{makeidx}			%enables to create an index with \makeindex in head, creates *.idx file
		\makeindex					%if index is required (after {makeidx})
	\usepackage[backend=biber]{biblatex}	%enables to print bibliography with \printbibliography, needs \bibliography{bib files}
		\bibliography{numerik_script.bib}					%loads .bib-file for biber-bibliography after {biblatex}
%
%..................................................
%%(nicer)tables
	\usepackage{booktabs} 			%enables better spacing and lines in tables
%	\usepackage{multirow}			%allows option multirow for tables (centers text vertically)
%	\usepackage{multicol}			%allows option multicol for tables
%	\usepackage{tabularx}  			%table with extendable X-column
%	\usepackage{tabulary}			%table-width matches content
%
%..................................................
%%(nicer) footnotes and marginpars
%	\usepackage[outer=4.5cm, marginparwidth=5cm]{geometry}		%provides commands to manipulate the dimension of several elements; needed for marginnote (outer is space width for margins)
%	\usepackage{marginnote}				%more individual marginpars with \marginnote{text}[width]; geometry package needed
%	\usepackage{todonotes}				%fancy styled (colored etc.) marginpars with \todo[options]{text}
%	\usepackage{footmisc}				%nicer footnotes
%
%..................................................

%strike text out
\usepackage{ulem}



%hyperlinks
	\usepackage{hyperref}			%for links/hyperlinks, better load last!, options: colorlink=true (-> no boxes but colored hyperlinks), color of all link
%
%
%..................................................
%fonts
%	\usepackage{anyfontsize}		%changes the fontsize with \fontsize{size}{baselineskip}\selectfont to any size





%DEFINITIONS
%------------------------------------------------------------------------------------------------
%..................................................
%math (theorems)
%\theoremstyle{definition}
%\newtheorem{Def}{Definition}		%definition with \begin{Def}
%\newtheorem{Ax}[Def]{Axiom}		%axiom with \begin{Ax}
%\newtheorem{Satz}[Def]{Satz}		%theorem with \begin{Satz}
%\newtheorem{Prop}[Def]{Proposition}%proposition with \begin{Prop}
%\newtheorem{Lem}[Def]{Lemma}		%lemma with \begin{Lem}
%\newtheorem{Korr}[Def]{Korrolar}	%corollary with \begin{Korr}

%colors
%\definecolor{ashgrey}{rgb}{0.7, 0.75, 0.71}




%NEW COMMAMDS
%------------------------------------------------------------------------------------------------
%\newcommand{\<new command>}{<what it shall do>}
\newcommand{\R}{\mathds{R}}
\newcommand{\Ren}{\mathds{R}^{n}}
\newcommand{\Renn}{\mathds{R}^{n\times n}}
\newcommand{\Renm}{\mathds{R}^{n\times m}}
\newcommand{\Q}{\mathds{Q}}
\newcommand{\N}{\mathds{N}}
\newcommand{\Z}{\mathds{Z}}
\newcommand{\F}{\mathds{F}}
\newcommand{\K}{\mathcal{K}}

% for the three boxes to show float numbers
\newcommand{\floatbox}[3]{ %
	\begin{array}{|c|c|c|}
		\cline{1-3} 	
		#1 & #2 & #3\\
		\cline{1-3}
	\end{array}
	}

% equation numeration
\newcommand{\nn}[1]{\left\| #1 \right\|}
\newcommand{\sectione}[1]{\section{#1} \setcounter{equation}{0}}

% inserted sections a), b)
\newcommand{\extrasection}[2]{\vspace{1.5eM}\minisec{\Large\itshape \thesection #1 #2}\vspace{1eM}}

\usepackage{framed}

% Pseudocode

\newenvironment{pseudocode}[1]{ %
%	\FrameCommand{\framebox[#1]}
%	\renewcommand{\FrameCommand}[1]{\framebox[#1]{}}
		\begin{minipage}{#1}
			\begin{framed}
				\hspace*{1em}	
				\begin{minipage}{#1}
					\begin{tabbing}
						for~~\= for~~\= for~~\= for~~\= \kill
	}
	{ %
					\end{tabbing}
				\end{minipage}
				\hspace*{1em}
			\end{framed}
		\end{minipage}
	}
	

%STYLE SETTINGS
%------------------------------------------------------------------------------------------------
%\pagestyle{scrplain}				%set pagestyle to scrheadings or scrplain (instead of headings or plain)
%\clearscrplain						%clear old style (either scrplain odr scrheadings)
%\cfoot[<text for scrplain>]{<text for scrheadings>}	%any new settings for foots/ headings
%\allowdisplaybreaks              %allow multipage for equations
\renewcommand{\theequation}{\thesection.\arabic{equation}}




%************************************************************************************************
\begin{document}
	
%----------------------------------------------------------------------------------------------
%FRONTMATTER
%----------------------------------------------------------------------------------------------
\frontmatter	%for book only, part for title etc.

%TITLE(PAGE)
%	\titlehead{titlehead (free)}
	\title{Skript Numerik I}
	\subtitle{von Prof. Dr. Luise Blank im WS14/15}
%	\subject{type}
	\author{Gesina Schwalbe}
%	\date{date}
%	\extratitle{Schmutztitel}
%	\publishers{Verlag}
%	\uppertitleback{Titelrückseitenkopf }
%	\lowertitleback{Titelrückseitenfuß}
%	\dedication{Widmung}
%	\thanks{Fußnote}
\maketitle

%---------------------------------------------------
\tableofcontents



%----------------------------------------------------------------------------------------------
%ACTUAL CONTENT
%----------------------------------------------------------------------------------------------
\mainmatter		%for book only, part for main content of document


% VORWORT

\chapter*{Vorwort}
%---------------------------------------------------
%\tableofcontents
\subsubsection{Skriptfehler}
An alle, die gedenken dieses Skript zur Numerikvorlesung im WS2014/15 zu 
nutzen: \\
Es wird keinerlei Anspruch auf Richtigkeit, Vollständigkeit und auch sicher nicht Schönheit
(ich bin LaTeX-Anfänger) dieses Dokuments erhoben.

Ihr würdet mir aber unglaublich weiterhelfen, wenn ihr jede Anmerkung 
-- das kann alles, von groben inhaltlichen 
Fehlern über Rechtschreibkorrekturen bis hin zu Wünschen/Anregungen/Tipps zur Typografie, sein --
an mich weiterleitet!\\

Jegliche Anmerkungen bitte gleich und jederzeit an: \\

\framebox[0.8\linewidth]{ %
	\bfseries \large
	
	gesina.schwalbe@stud.uni-regensburg.de}

\hspace{1cm}

 oder auch an \textbf{gesina.schwalbe@googlemail.com}


\subsubsection{Copyright}
Was das Rechtliche angeht bitte beachten: \\
Urheber dieses Skriptes ist Prof. Dr. Luise Blank. \\
Dies ist nur eine genehmigte Vorlesungsmitschrift und unterliegt dem deutschen
Urheberrecht, jegliche nicht rein private Verwendung muss demnach vorher mit
Frau Blank abgesprochen werden.\\


\subsubsection{Bilder oder \enquote{IMAGE MISSING}}
Leider habe ich mich bisher noch nicht in die (ordentliche) Grafikerstellung in 
LaTeX-Dokumenten eingearbeitet, weshalb das Skript erstmal nicht mit 
Grafiken dienen kann -- die Fehlstellen sind mit \enquote{IMAGE MISSING} markiert. \\
Wenn ihr gerne Veranschaulichungen haben möchtet, könnt ihr jederzeit \textbf{die
	entsprechenden Bilder an mich schicken}, bitte mit Kapitelangabe. (Oder mich explizit darum bitten, dass ich mich übergangsweise um das Einscannen, Bearbeiten etc. kümmere).
Dann werden sie an die entsprechenden Stellen eingebunden.


\subsubsection{Erscheinungsdatum}
Ich werde mich bemühen, das Skript jeweils am Vorlesungstag zumindest in 
unverbesserter Form online zu stellen, so dass v.a. diejenigen, die die Vorlesung
nicht besuchen können, einen Überblick über den Stoff bekommen.

Innerhalb einer Woche sollte das Skript aktuell sein.


\subsubsection{Anfangsteil }
Nachdem ich nicht seit Semesterbeginn mittexe, fehlt noch ein Großteil des
ersten Kapitels. Ich hoffe, es bis Mitte des Semesters nachholen zu können, 
aber keine Garantie. \\
Ziel ist, dass es vor der Vorlesungszeit integriert ist.





\marginpar{06.10.2014}
\chapter{Einführung}

\chapter{Lineare Gleichungssysteme: Direkte Methoden}
\label{2.1}
Sei $ A \in \R^{n\times n}$, $b \in \R^n$. Gesucht ist $x\in \R^n$ mit 
\begin{gather*}
	A\cdot x = b
\end{gather*}
Weitere Voraussetzungen sind die Existenz und Eindeutigkeit einer Lösung.
Bemerkung:
\begin{itemize}
	\item Ein verlässlicher Lösungsalgorithmus überprüft dies und behandelt alle Fälle. 
	\item Die Cramersche Regel ist ineffizient (s. Einführung).
	\item Das Inverse für $x=A^{-1}\cdot b$ aufzustellen ist ebenso ineffizient, denn es ist keine Lösung für alle $b\in \R^n$ verlangt und der Algorithmus wird evtl. instabil aufgrund vieler Operationen.
	\item [$\Rightarrow$] Invertieren von Matrizen vermeiden!!
	\item [$\Rightarrow$] Lösen des Linearen Gleichungssystems!!
\end{itemize}

\sectione{Gaußsches Eliminationsverfahren} \label{2.1} \index{Gaußsches Eliminationsverfahren}\index{Dreieckszerlegung}
Das Verfahren wurde 1809 von Friedrich Gauß, 1759 von Josepf Louis Lagrange beschrieben und war seit dem 1. Jhd. v. Chr. in China bekannt.

\subsection{Vorwärtselimination} \label{2.1.1}\index{Vorwärtselimination}\index{Vorwärtssubstitution}
Das Gaußverfahren gilt der Lösung eines linearen Gleichungssystems der Form
\begin{align*}
	Ax &= b
\end{align*}
mit $A=(a_{ij})_{i,j \leq n} \in K^{n\times n}$ Matrix und $b=(b_i)_{i\leq n} \in K^n$ Vektor.\\
Der zugehörige Algorithmus sieht folgendermaßen aus:
\begin{gather*}
	\begin{array}{ccccccccc}
	a_{11}x_1 &+& a_{12}x_2 &+& \cdots &+& a_{1n}x_n & = & b_1 ~~\\
	a_{21}x_1 &+& a_{22}x_2 &+& \cdots &+& a_{2n}x_n & = & b_2 \\
	\vdots         &&        \vdots     &&              &&   \vdots       &    & \vdots \\
	a_{n1}x_1 &+& a_{n2}x_2 &+& \cdots &+& a_{nn}x_n & = & b_n \\\\
	&&&& \Downarrow &&&& 
	\end{array} \\
\quad 	(\text{i-te Zeile}) - (\text{1. Zeile})\cdot \frac{a_{i1}}{a_{11}} \Rightarrow a_{i1}=0\\
\begin{array}{ccccccccc}
&&&& \Downarrow &&&&  \\\\
a_{11}x_1 &+& a_{12}x_2 &+& \cdots &+& a_{1n}x_n & = & b_1 \\
				  &+& a_{22}^{(1)}x_2 &+& \cdots &+& a_{2n}^{(1)}x_n & = & b_2^{(1)} \\
				 &&        \vdots     &&              &&   \vdots       &    & \vdots \\
														&& && && a_{nn}^{(1)}x_n & = & b_n^{(1)} \\\\
&&&& \Downarrow &&&&\\
&&&& \vdots &&&&
\end{array} 
\end{gather*}
mit
\begin{align*}
	a_{ij}^{(1)} &= a_{ij}-a_{1j}\cdot \frac{a_{i1}}{a_{11}} & \text{für }i,j = 2, \cdots, n \\
	b_i^{(1)}      &= b_i- b_1\cdot \frac{a_{i1}}{a_{11}}        & \text{für }i = 2, \cdots, n 
\end{align*}
\marginpar{08.10.2014}
In jedem Schritt werden die Einträge der $k$-ten Spalte analog unterhalb der Diagonalen (also $k=1, \cdots, n-1$) eliminiert:
\begin{align*}
	(\text{$i$-te Zeile})- (\text{$k$-te Zeile})\cdot\frac{a_{ik}}{a_{kk}} && \text{für } i=k+1, \cdots ,n 
\end{align*}
Die Reihe 
\begin{gather*}
			A \rightarrow A^{(1)} \rightarrow A^{(2)} \rightarrow \dotsm \rightarrow A^{(n-1)}
\end{gather*}
wird bis zum $n$-ten Schritt fortgeführt, d.h. bis eine obere Dreiecksgestalt eintritt:
\begin{align}
\nonumber
\underbrace{	\begin{pmatrix}
	a_{11} & \dotsm & \dotsm & a_{1n} \\
	             & a_{22}^{(1)} & \dotsm & a_{2n}^{(1)} \\
	             &&              \ddots  &  \vdots \\
	   0        && &                             a_{nn}^{(n-1)}
	\end{pmatrix}}_{\coloneqq R}
	\cdot
	\begin{pmatrix}
		x_1 \\
		x_2 \\
		\vdots \\
		x_n
	\end{pmatrix}
	& =
	\underbrace{\begin{pmatrix}
		b_1 \\
		b_2^{(1)} \\
		\vdots \\
		b_n^{(n-1)}
	\end{pmatrix}}_{\coloneqq z} \\
Rx &= z 	\label{II.1.1} 
\end{align}
wobei für  $i=k+1, \cdots ,n$ die Einträge wie folgt aussehen:
\begin{align}	
	l_{ik} &\coloneqq \frac{a_{ik}^{(k-1)}}{a_{kk}^{(k-1)}} \label{II.1.2} \\
	a_{ij}^{(k)} &= a_{ij}^{(k-1)} - a_{kj}^{(k-1)}\cdot l_{ik} \label{II.1.3}
				 & \text{für } j=k+1, \cdots , n\\ %
	b_i^{(k)} &= b_i^{(k-1)} -b_k^{(k-1)} \cdot   l_{ik}\label{II.1.4}\index{Vorwärtssubstitution}
\end{align}
Dieser Prozess wird \textbf{Vorwärtselimination} genannt.\\

Der zugehörige Algorithmus ist:

\begin{pseudocode}{0.5\linewidth}
	\textbf{for} $ k = 1, \dots , n-1$\\
	|~	\> \textbf{for} $i = k + 1, \dots , n$ \\
	|~	\> |~\> $l_{ik} = a_{ik} /a_{kk}$\\
	|~	\> |~\> \textbf{for} $j = k + 1, \dots , n$ \\
	|~	\> |~\>|~\> $a_{ij} = a_{ij} - l_{ik} a_{kj} $\\
	|~	\> |~\> \textbf{end}\\
	|~	\> |~\> $b_i = b_i -  l_{ik} b_k $\\
	|~	\> \textbf{end} \\
\textbf{end}
\end{pseudocode}

\subsection{Rückwärtselimination}\label{2.1.2}
Für die Lösung des Gleichungssystems ist dann noch die \textbf{Rückwärtssubstitution} \index{Rückwärtssubstitution} nötig:
\begin{align}
	x_n &= \frac{b_n^{(n-1)}}{a_{nn}^{(n-1)}} \label{II.1.5} \\
	x_{n-1} &=  \frac{b_{n-1}^{(n-2)}-a_{n-1,n}^{(n-1)}\cdot x_n}{a_{(n-1)(n-1)}^{(n-2)}} \label{II.1.6} \\
	x_k &= \frac{b_k^{(k-1)}-\sum_{j=k+1}^{n}a_{kj}^{(k-1)}x_j}{a_{kk}^{(k-1)}} \label{II.1.7}
\end{align}

Als Algorithmus:

\begin{pseudocode}{0.5\linewidth}
	\textbf{for} $k = n, n -1, \dots , 1$ \\
	|~		\>$x_k = b_k$ \\
	|~		\>\textbf{for} $j = k + 1, \dots , n$ \\
	|~		\>|~	$x_k = x_k - a_{kj}x_j$ \\
	|~		\>\textbf{end} \\
	|~		\>$x_k = x_k /a_{kk}$ \\
	\textbf{end}
\end{pseudocode}

\subsection{Vorsicht}
	Algorithmen \ref{2.1.1} und \ref{2.1.2} sind nur ausführbar, falls für die sog. \textbf{Pivotelemente $\mathbf{a_{kk}^{(k-1)}}$ } \index{Pivotelement} gilt:
	\begin{gather*}
			a_{kk}^{(k-1)} \neq 0 \quad   \text{für } k=1, \cdots , n
	\end{gather*}
	Dies ist auch für invertierbare Matrizen nicht immer gewährleistet.
	
\subsection{Weitere algorithmische Anmerkungen}	\label{2.1.4}
Matrix $A$ und Vektor $b$ sollten möglichst \textbf{nie} überschrieben werden! (Stattdessen kann eine Kopie überschrieben werden.) \\
Das Aufstellen von $A$ und $b$ ist bei manchen Anwendungen das teuerste, sie gehen sonst verloren. In \ref{2.1.1} wird das obere Dreieck von $A$ überschrieben. Dies ist möglich, da in \eqref{II.1.3} nur die Zeilen $k+1, \cdots, n$ mithilfe der $k$-ten bearbeitet werden. Am Ende steht $R$ im oberen Dreieck von $A$ und $z$ in $b$. \\
Die $l_{ik}$ werden spaltenweise berechnet und können daher anstelle der entsprechenden Nullen (in der Kopie) von $A$ gespeichert werden, d.h.:
\begin{gather}
	\widetilde{L} \coloneqq (l_{ik})  \label{II.1.8}
\end{gather}
und $R$ werden sukzessive in A geschrieben. \\
IMAGE~MISSING \\
Der Vektor $z$ und anschließend der Lösungsvektor $x$ kann in (eine Kopie von) $b$ geschrieben werden.
Wird eine neue rechte Seite $b$ betrachtet, muss \ref{II.1.1} nicht komplett neu ausgeführt werden, da sich $\widetilde{L}$ nicht ändert. Es reicht \ref{II.1.4} zu wiederholen. \\
IMAGE~MISSING \\

\subsection{Dreieckszerlegung} \label{2.1.5} \index{Dreieckszerlegung}
Die Dreieckszerlegung einer Matrix $A$ entspricht dem Verfahren aus \ref{2.1.1}, nur ohne die Zeile \eqref{II.1.4}.

\subsection{Vorwärtssubstitution} \index{Vorwärtssubstitution}
Die Vorwärtssubstitution entspricht der in \ref{2.1.4} bzw. dem Verfahren aus \ref{2.1.1} ohne die Bestimmung von $l_{ik}$ und $R$, also nur Schritt \eqref{II.1.4}.

\subsection{Gauß-Elemination zur Lösung von $Ax=b$}\index{Gauß-Eleminator}
\begin{framed}
	\begin{enumerate}[1]
		\item Dreieckszerlegung
		\item Vorwärtssubstitution        $\quad b_i^{(k)} = b_i^{(k-1)} -b_k^{(k-1)} \cdot   l_{ik} $
		\item Rückwärtssubstitution      $\quad x_k = \frac{b_k^{(k-1)}-\sum_{j=k+1}^{n}a_{kj}^{(k-1)}x_j}{a_{kk}^{(k-1)}}$
	\end{enumerate}
\end{framed}

\subsection{Rechenaufwand gezählt in \enquote{flops}} \index{flops}\index{floating point operations}\index{Rechenaufwand}
\textbf{\enquote{flops} }= \textbf{f}loating \textbf{p}oint \textbf{op}eration\textbf{s} \\
\begin{enumerate}
	\item[\textbf{1.}] \textbf{Dreieckszerlegung} 
			\begin{tabbing}
				für $j=k+1, \dots, n\quad$ \= noch ein wenig text danach \kill
				für $j=k+1, \dots, n$ \> 1 Addition, 1 Multiplikation für $a_{ij}$ \\
				für $i=k+1, \dots, n$ \> 1 Division zusätzl. für $ l_{ik}$
			\end{tabbing}
			Dies ist je für $k=1, \dots, n-1$, also ist die Zahl an Additionen und Multiplikationen
			\begin{align*}
				\sum_{k=1}^{n-1}(n-k)^2 &= \sum_{k=1}^{n-1}k^2 \\
				 											&= \frac{(n-1)n(2n-1)}{6} \\
				 									        &= \frac{2n^3-3n^2+n}{6}\, .
			\end{align*}
			Für große $n$ sind das etwa $\frac{n^3}{3}$ Additionen und Multiplikationen und
			\begin{gather*}
				\sum_{k=1}^{n-1} (n-k) = \frac{n^2-n}{2} \approx \frac{n^2}{n}
			\end{gather*}
			Divisionen. \\
			Damit ergibt sich eine Gesamtanzahl an flops von
			\begin{gather*}
				2\cdot\frac{2n^3-3n^2+n}{6} + \frac{n^2-n}{2} 
					= \frac{2}{3} n^3 - \frac{1}{2}n^2 - \frac{1}{6} n
					 \approx \frac{2}{3}n^2
			\end{gather*}
			für große $n$.
			
	\item[\textbf{2.}] \textbf{Vorwärts- bzw. Rückwärtssubstitution}  \\
			Hier ergeben sich je
			\begin{gather*}
				\sum_{k=1}^{n-1} (n-k) = \frac{n^2-n}{2} \approx \frac{n^2}{2}
			\end{gather*}
			Multiplikationen und Additionen sowie 
			$n$ Divisionen für die Rückwärtssubstitution und damit insgesamt \begin{gather*}n^2+n\end{gather*} flops.	
\end{enumerate}
\paragraph{Zusammenfassung}~ \\
Die Dreieckszerlegung benötigt $\mathcal{O}(n^3)$ flops und 
die Vorwärts- bzw. Rückwärtssubstitution $\mathcal{O}(n^2)$ flops.



\subsection{Definition: Landau-Symbole} \index{Landau-Symbole}
MISSING

\subsection{Allgemeines zur Aufwandsbetrachtung}\index{Rechenaufwand}
MISSING

\marginpar{13.10.2014}
\subsection{Formalisieren des Gauß-Algorithmus} \index{Gaußsches Eliminationsverfahren} \index{LR-Zerlegung}


MISSING

\subsection{Lemma (Eigenschaften der $L_k$-Matrizen)} \label{2.1.12} \index{Frobeniusmatrix}
\begin{enumerate}[1.]
	\item $L_k$ ist eine Frobeniusmatrix, d.h. sie unterscheidet sich höchstens
			 in einer Spalte von der Einheitsmatrix $I$.
	\item $L_k^{-1} = I + l_ke_{k}^T$
	\item Es gilt:
			\begin{align}
				L &= L_1^{-1} \cdot \dotsc \cdot L_{n-1}^{-1}  
				\label{II.1.13}
				\\ \nonumber
					& = I + \sum_{i=1}^{n-1} l_i e_i^T \\ \nonumber
					&= \begin{pmatrix}
						1 && 0 ~ \\
						&\ddots& \\
						~l_{ij} && ~1
					\end{pmatrix} \\ \nonumber
					&= I+ \widetilde{L}
			\end{align}
			
			Hiermit folgt: 
\end{enumerate}

\subsection{Satz (LR- oder LU-Zerlegung)} \index{LR-Zerlegung}\index{LU-Zerlegung}
Das obige Verfahren (\eqref{II.1.2} und \eqref{II.1.3}) erzeugt unter der Voraussetzung
von nicht-nullwertigen Pivotelementen eine Faktorisierung
\begin{align*}
	A= L\cdot R && IMAGE~MISSING 
\end{align*}
wobei $R$ eine obere Dreiecksmatrix und $L$ eine untere, normierte Dreiecksmatrix ist,
d.h. für $i=1, \cdots , n$ gilt $l_{ii}= 1$. \\
Weiterhin existiert zu jeder regulären Matrix höchstens eine solche Zerlegung.

\paragraph{Beweis} ~ \\
MISSING
\index{Verfahren von Crout}

\sectione{Gaußsches Eliminationsverfahren mit Pivotisierung}
\paragraph{Beispiel} Die Matrix $A= \begin{pmatrix}0&1\\1&0\end{pmatrix} $
ist invertierbar, aber die Gauß-Elimination versagt. Permutiere also die erste mit der
zweiten Zeile und der Algorithmus wird anwendbar.

\paragraph{Allgemein} Vermeide die Division durch betragsmäßig kleine Zahlen! 


\subsection{Spaltenpivotisierung (=partielle/ halbmaximale Pivotisierung)} \index{Pivotisierung!Spalten-}\index{Pivotisierung}\index{Pivotisierung!halbmaximale}\index{Pivotisierung!partielle}
Im k-ten  Eliminationsschritt ist \\

IMAGE~MISSING (s. Folien) \\

MISSING

\subsection{Bemerkung}
\begin{enumerate}[a)]
	\item Hiermit gilt $|l_{jk} \ll 1$.
	\item Anstelle von Spaltenpivotisierung kann eine \textbf{Zeilenpivotisierung}\index{Pivotisierung!Zeilen-}
	durchgeführt werden.
	Welche günstiger (in cpu-time) ist, hängt von der Rechnerarchitektur und
	der damit zusammenhängenden Umsetzung des Gauß-Algorithmus ab.\\
	(Beispielsweise greifen Vektorrechner entweder auf die gesamte Spalte
	oder auf die gesamte Zeile einer Matrix zu und bevorzugen dementsprechend
	Operationen spalten- bzw. zeilenweise.)
	\item Der Aufwand enthält (bis auf $|\cdot |$) keine Rechenoperationen (flops),
	aber $\mathcal{O}(n^2)$ Vergleiche und Vertauschungen.
	\item Eine \textbf{vollständige Pivotsuche}\index{Pivotisierung!vollständige} sucht das betragsmäßig größte Element der gesamten Restmatrix und benötigt $\mathcal{O}(n^3) $ Vergleiche
	(sie wird so gut wie nie angewendet).
\end{enumerate}

Damit die LR-Zerlegung unabhängig von der rechten Seite erstellt werden kann, müssen die Permutationen gespeichert werden.
Hierfür verwendet man  einen sog. \textbf{Permutationsvektor} $\Pi$, wobei
\begin{gather*}
	\Pi^{(k-1)}(r) = s
\end{gather*}
bedeutet, dass nach dem $(k-1)$-ten Eliminationsschritt in der $r$-ten Zeile
von $A^{(k-1)}$ die $s$-te bearbeitete Zeile von $A$ steht, also
\begin{align*}
	\Pi^{(k)}(k) &= \Pi^{(k-1)}(p) \\
	\Pi^{(k)}(p) &= \Pi^{(k-1)}(k)  \quad \text{und entsprechend} \\
	\scriptsize \Pi^{(k)} (i) &= \Pi^{(k)} (i) \quad \text{ für }  i\neq k,p 
\end{align*}
Für die \textbf{Permutationsmatrix}\index{Permutationsmatrix}  
\begin{align*}
	P_{\Pi}&=(e_{\Pi(1)}, \dots , e_{\Pi(n)}) 
	 &&
		 e_j\coloneqq \begin{pmatrix}
		 						 0\\\vdots\\0\\
		 						 1 \\
		 						 0\\\vdots\\0 
		 					  \end{pmatrix}
		 					   \begin{array}{l}
			 					   \\\\
			 					   \shortleftarrow \text{j-te Stelle}\\
			 					   \\\\
		 					   \end{array} \\
\text{mit }	\quad PA&= LR
\end{align*}
gilt
\begin{gather*}
	P^{-1} = P^T
\end{gather*}
und
\begin{gather*}
	\det P_{\Pi} = sign(\Pi) =  \begin{cases}
		+1&  \text{falls }\Pi \text{ von gerader}\\
		-1 &  \text{falls } \Pi \text{ von ungerader}
	\end{cases} \text{ Anzahl an Transpositionen erzeugt wird}
\end{gather*}


\subsection{Gauß-Elimination mit Spaltenpivotisierung}
Der zugehörige Algorithmus zur Spaltenpivotisierung ist: \\

\begin{pseudocode}{0.9\linewidth}
	$\pi(1 : n) = [1 : n] $  \\
	\textbf{for} $k = 1, \dots, n-1$ \\
	~|	  \> bestimme Pivotzeile $p$, so dass\\
	~|	  \>\>	$|a_{pk} | = \max\{|a_{jk} | , j = k, . . . , n\}$ \\
	~|	  \> $\pi(k) \leftrightarrows\footnotemark \pi(p)$ \\
	~|	  \> $A(k, 1 : n) \leftrightarrows A(p, 1 : n)$ \\
	~|	  \> \textbf{if} $a_{kk} \neq 0$ \\
	~|	  	   \>~|\> $zeile = [k + 1 : n]$ \\
	~|	  	   \>~|\> $A(zeile, k) = A(zeile, k)/a_{kk}$ \\
	~|	  	   \>~|\> $A(zeile, zeile) = A(zeile, zeile) - A(zeile, k)A(k, zeile)$\\
	~|	  \> \textbf{else} \\
	~|	  	   \>~|\> \enquote{$A$ ist singulär}\\
	~|	  \> \textbf{end}\\
	\textbf{end}
	\footnotetext{$\leftrightarrows$ bedeutet \enquote{vertausche mit}}
\end{pseudocode}\\

\subsection{Satz: Dreieckszerlegung mit Permutationsmatrix} \label{2.2.4} 
	Für jede invertierbare Matrix $A$ existiert eine Permutationsmatrix $P$,
	so dass eine Dreieckszerlegung
	\begin{gather*}
		PA = LR
	\end{gather*}
	existiert.
	$P$ kann so gewählt werden, dass alle Elemente von $L$ betragsmäßig kleiner oder gleich
	1 sind, d.h.
	\begin{gather*}
		|l_{ij}| \leq 1\quad \forall i, j
	\end{gather*}
	
	
\paragraph{Beweis}~ \\
	Da $\det A \neq 0$ ist, existiert eine Transposition $\tau_1$, s.d. 
	\begin{gather*}a_{11}^{(1)}= a_{\tau_1, 1} \neq 0 \end{gather*}
	und
	\begin{gather*}
		| a_{\tau_1, 1} | \geq |a_{i1}| 
		\quad \forall i=1, \cdots, n \, . 
	\end{gather*}
	Wir erhalten damit
	\begin{gather*}
		L_1P_{\tau_1} \cdot A = A^{(1)} = \begin{pmatrix}
																	a_{11}^{(1)} && \cdots ~&  ~\\
																	0 \\
																	\vdots && B^{(1)} \\
																	0
																\end{pmatrix}
	\end{gather*}
	und alle Elemente von $L_1 $ sind betragsmäßig kleiner oder gleich 1 sowie
	$\det L_1 = 1 $. \\
	Daraus folgt
	\begin{align*}
		\det B^(1) & = \frac{1}{\underbrace{a_{11}^{(1)}}_{\neq 0}} \cdot \det A^{(1)} \\
						 & = \frac{1}{a_{\tau_1, 1}^{(1)}}\cdot \det (L_1) \cdot \det (A) \\
						 & \neq 0 \, .
	\end{align*}
	Also ist $B^(1)$ invertierbar. \\
	
	Induktiv erhalten wir dann
	\begin{gather*}
		R = A^{(n-1)} = L_{n-1}P_{\tau_{n-1}} \cdot \dotsc \cdot L_1 P_{\tau_1} \cdot A
	\end{gather*}\\
\marginpar{15.10.2014}
	Da $\tau_i$ nur zwei Zahlen $\geq i $ vertauscht, ist
	\begin{align*}\index{Vorwärtselimination}
		\Pi_i  &\coloneqq \tau_{n-1} \circ \dots \circ \tau_i \quad\text{ für } i=1,\dots (n-1) 
		\end{align*}
	eine Permutation der Zahlen $\{i,\dots, n\}$, d.h. insbesondere gilt:
	\begin{alignat*}{2}
		\Pi_i(j)&=j  & \quad &\text{ für } j=1,\dots,(i-1) \\
		\Pi_i(j)&\in \{i, \dots, n\} & &\text{~für~}j=i,\dots, n\,. 
	\end{alignat*}
	\begin{align*}
		P _{\Pi_{i+1}}  &= (e_1, \dotsc e_i, e_{\Pi_{i+1}(i+1)}, \dotsc, e_{\Pi_{i+1}(n)}) 
								= \begin{pmatrix}
										I_i & 0 \\
										0 & P_{\sigma}
								\end{pmatrix}
	\end{align*}
	Damit folgt:
	\begin{align*}
	P_{\Pi_(i+1)}\cdot L_i\cdot P_{\Pi_{i+1}}^{-1}  &= 
													P_{\Pi_{i+1}} \cdot \left(\begin{array}{ccc|ccc}
																						& I_i & && 0 & \\
																						\cline{1-6}
																						&     & -l_{i+1, i} & & & \\
																						&  0 &  \vdots      & & I_{n-i} &\\
																						&     & -l_{n, i} & &  & 
																					\end{array}\right)
																			\cdot \begin{pmatrix}
																					I_i & 0 \\
																					0 & P_{\sigma}^{-1}
																			\end{pmatrix}\\
	 &= \begin{pmatrix}
				 I_i & 0 \\
				 0 & P_{\sigma}
			 \end{pmatrix} \cdot   \l\cdot  \cdot  \left(\begin{array}{ccc|ccc}
													 & I_i & && 0 & \\
													 \cline{1-6}
												\cdot  	 &     & -l_{i+1, i} & & & \\
													 &  0 &  \vdots      & & P_{\sigma}^{-1} &\\
													 &     & -l_{n, i} & &  & 
												\end{array}\right) \\
	 &= \left(\begin{array}{ccc|ccc}
				 & I_i & && 0 & \\
				 \cline{1-6}
				 &     & -l_{\Pi_{i+1}(i+1), i} & & & \\
				 &  0 &  \vdots      & & I_{n-i} &\\
				 &     & -l_{\Pi_{i+1}(n), i} & &  & 
		 \end{array}\right) \\
	 &= I - (P_{\Pi_{i+1}} l_i)e_i^T\\
	 &\eqqcolon \widehat{L}_i
	\end{align*}
	und
	\begin{align*}		R =&\, L_{n-1}\\
					&\cdot (P_{\tau_{n-1}}L_{n-2}P_{\tau_{n-1}}^{-1})\\
		&				\cdot (P_{\tau_{n-1}}P_{\tau_{n-2}}L_{n-2}P_{\tau_{n-2}}^{-1}P_{\tau_{n-1}}^{-1})\\
		&\; \vdots \\
		&		 \cdot (P_{\tau_{n-1}}\dotsm P_{\tau_{1}}L_{1}P_{\tau_{1}}\dotsm P_{\tau_{n-1}}) \cdot A\\
=&\,L_{n-1}\widehat{L}_{n-2}\dotsm\widehat{L}_1P_{\Pi_{1}}\cdot A
\end{align*}
Nach Lemma \ref{2.1.12} gilt daher, es existiert eine Permutation $\Pi_{1}$ mit
\begin{gather*}
	P_{\Pi_1}\cdot A = LR ,
	\end{gather*}
wobei $R$ obere Dreiecksgestalt hat und
\begin{align*}
		L  &=  \begin{pmatrix}
						1 && & 0\\
						l_{\Pi_2(2),1} & \ddots & \\
						\vdots &            \ddots &  1\\
						l_{\Pi_n(n),1}& \dotsm &  l_{\Pi_n(n),n-1} & 1 
					\end{pmatrix} 
					& \text{mit } |l_{ij}| \leq 1 
\end{align*}
gilt.

\subsection{Lösen eines Gleichungssystems $Ax=b$} \label{2.2.5}
\subsection{Bemerkungen}
\subsection{Beispiel zur Pivotisierung}


\marginpar{15.10.2014}
\chapter{Fehleranalyse} \index{Fehler}\label{3}
%
\begin{align*}
	\overset{x+\epsilon \text{ statt } x}{\framebox[3cm]{Eingabe}} \longrightarrow 
	\overset{\underset{\text{\tiny (z.B. durch Rundung)}}{\widetilde{f} \text{ statt } f}}{\framebox[3cm]{Algorithmus}} \longrightarrow
	\overset{\widetilde{f}(x+\epsilon) \text{ statt } f(x)}{\framebox[3cm]{Resultat\phantom{g}}}
\end{align*}\\

Bei der Fehleranalyse liegt das Hauptaugenmerk auf
\begin{itemize}
	\item[] \textbf{Eingabefehler}\\ z.B.Rundungsfehler, Fehler in Messdaten, Fehler im Modell (falsche Parameter)
	\item[] \textbf{Fehler im Algorithmus} \\ z.B. Rundungsfehler durch Rechenoperationen, Approximationen \\
	 (z.B. Ableitung durch Differenzenquotient oder die Berechnung von Sinus durch abgebrochene Reihenentwicklung)
	\\
	\item[\textit{1. Frage}] Wie wirken sich Eingabefehler auf das Resultat unabhängig vom gewählten Algorithmus aus?
		\item[\textit{2. Frage}]Wie wirken sich (Rundungs-)Fehler des Algorithmus aus?\\
											Und wie verstärkt der Algorithmus Eingabefehler?
\end{itemize}


\sectione{Zahlendarstellung und Rundungsfehler} \label{3.1} \index{Fehler} \index{Rundungsfehler}\index{Zahlendarstellung}
Auf (Digital-)Rechnern können nur endlich viele Zahlen realisiert werden. \\
Die wichtigsten Typen sind: 
\begin{itemize}
	\item \textbf{ganze Zahlen}  (integer)\index{integer}:
					\begin{align*}
						 z&=\pm \sum_{i=0}^{m}z_i\beta_i & \text{mit }
						 \begin{array}{l@{\,}l}
							 \beta &= \text{Basis des Zahlensystems (oft $\beta=2$)} \\
							 z_i &\in \{0, \cdots \beta-1\}
						 \end{array}
						\end{align*}
	\item \textbf{Gleitpunktzahlen} (floating point) \index{floating point}
\end{itemize}

\subsection{Definition: Gleitkommazahl} \label{3.1.1} \index{Gleitkommazahl}
Eine Zahl $x\in\Q$ mit einer Darstellung
\begin{align*}
	x&=\sigma \cdot(a_1 . a_2 \cdots a_t)_{\beta}\cdot \beta^e 
	 = \sigma\beta^e \cdot \sum_{\nu=1}^{t}a_\nu \beta^{-\nu+1}\\\\
&\quad\begin{array}{ll}
	\beta\in\N & \text{Basis des Zahlensystems}\index{Basis}\\
	\sigma \in\{\pm 1\} &\text{Vorzeichen} \\
	m = (a_1 . a_2 \cdots a_t)_{\beta} &\text{Mantisse}\index{Mantisse}\\
	\phantom{m}= \sum_{\nu=1}^{t}a_\nu \beta^{-\nu+1} \\
	a_i \in\{0,\cdots , \beta-1\}&\text{Ziffern der Mantisse}\\
	t\in\N&\text{Mantissenlänge} \\
	e\in\Z &\text{mit }e_{min}\leq e \leq e_{max} \text{ Exponent}
\end{array}
\end{align*}
heißt \textbf{Gleitkommazahl} mit $t$ Stellen und Exponent $e$ zur Basis $b$. \\
Ist $a_1\neq 0$, so heißt $x$ \textbf{normalisierte Gleitkommazahl}\index{normalisierte Gleitkommazahl}

\subsection{Bemerkung} \label{3.1.2}
\begin{enumerate}[a)]
	\item 0 ist keine normalisierte Gleitkommazahl, da $a_1 =  0$ ist.
	\item $a_1\neq 0$ stellt sicher, dass die Gleitkommadarstellung eindeutig ist.
	\item In der Praxis werden auch nicht-normalisierte Darstellungen verwendet.
	\item Heutige Rechner verwenden meist $\beta =2$, aber auch $\beta=8, \beta=16$.
\end{enumerate}

\subsection{Beispiel} \label{3.1.3}
 bit-Darstellung nach IEEE-Standard 754 von floating point numbers \\
 Sei die Basis $\beta=2$.
 
\begin{tabular}{l@{}cccc@{}}
				& Speicherplatz & $t$ & $e_{min}$ & $e_{max}$ \\
				\cmidrule{2-5}
	einfache Genauigkeit (float) \index{floating point} & 32bits = 4Bytes & 24 &-126 & 127 \\
	doppelte  Genauigkeit (double)~~\index{double} & 64bits = 8Bytes& 52 & -1022 & 1023
\end{tabular}\\

Darstellung im Rechner (Bitmuster) für float:
\begin{gather*}
\floatbox{s}{b_0\cdots b_7}{a_2\cdots\cdots a_{24}}\\
\text{(Da $a_1\neq 0$, also $a_1=1$ gilt, wird $a_1$ nicht gespeichert)}
\end{gather*}

Interpretation ($s,b,a_i\in\{0,1\} \forall i$)
\begin{itemize}
		\item$s$ Vorzeichenbit: $\quad \sigma=(-1)^s 
										\Rightarrow \begin{array}{l}
																\sigma(0)=1 \\
																\sigma(1)=-1
															\end{array} $
		\item $b=\sum_{i=0}^{7}b_i\cdot2^i \in \{1, \cdots, 254\}$ speichert den Exponenten mit \\
							$ \quad e = b-\underbrace{127}_\text{Basiswert}$ (kein Vorzeichen nötig) \\
							Beachte: $b_0=\cdots=b_7=1$ sowie $b_0=\cdots=b_7=0$ sind bis auf Ausnahmen keine gültigen Exponenten
		\item $m=(a_1.a_2\cdots a_{24})=1+\sum_{\nu=2}^{24}a_{\nu}2^{1-\nu}$ stellt die Mantisse dar, $a_1=1$ wird nicht abgespeichert.
		\item Besondere Zahlen per Konvention:
		\begin{itemize}
			\item[$x=0$:] $s$ bel., $b=0$, $m=1 \quad \floatbox{s}{0\cdots0}{0\cdots0}$
			\item[$x=\pm\infty$:]  $s$ bel., $b=255$, $m=1  \quad \floatbox{s}{1\cdots1}{0\cdots0}$
			\item[$x=$NaN] $s$ bel., $b=255$, $m\neq 1$
			\item[$x=(-1)^s$] $s$ bel., $b=0$, $m\neq 1$ und x hat die Form $x=(0+\sum_{\nu=2}^{24}a_{\nu}\cdot 2^{1-\nu})\cdot 2^{126}$ (\enquote{denormalized} number)
		\end{itemize}
\end{itemize}

  
% \marginpar{20.10.2014}
\begin{align*}
\intertext{Betragsmäßig \textbf{größte Zahl}:}
	\floatbox{0}{01\cdots 1}{ 1\cdots\cdots 1} && 
	 x_{max} = (2-2^{-23})\cdot 2^{127}  & \approx 3,4 \cdot 10^{38}
\intertext{Betragsmäßig \textbf{kleinste Zahl}:}
	\floatbox{0}{0\cdots 0}{ 0\cdots\cdots 01} && 
	x_{min} = (2-2^{-23})\cdot 2^{-126} = 2^{-149}  & \approx 1,4 \cdot 10^{-45}
\end{align*}

\subsection{Verteilung der Maschinenzahlen} \label{3.1.4}
ungleichmäßig im Dezimalsystem, z. B.
\begin{align*}
		x &= \pm a_1 . a_2 a_3 \cdot 2^e  & -2\leq & e\leq 1 & a_i & \in \{0,1\}  \\
		IMAGE~MISSING
\end{align*}
ist im Dualsystem gleichmäßig verteilt.

\subsection{Bezeichnungen} \label{3.1.5}
\begin{itemize}
	\item[\textbf{overflow}] es ergibt sich eine Zahl, die betragsmäßig größer ist als die größte maschinendarstellbare Zahl
	\item[\textbf{underflow}] entsprechend, betragsmäßig kleiner als die kleinste positive Zahl
\end{itemize}
Bsp.: overflow beim integer $b=e+127$
\begin{align*}
	\begin{array}{rrr@{}}
	b &= 254                                &11111110 \\
	   &+  \phantom{24}3 &00000011 \\
	   \cline{3-3} %
	 b+3 = 257 \text{ mod } 2^8  &=\phantom{24}1& \xout{1}00000001 
	\end{array}	  
\end{align*}


\subsection{Rundungsfehler} \label{3.1.6}
Habe $x\in \R $ die normalisierte Darstellung
\begin{align*}
	x &= \sigma \cdot \beta^e (\sum_{\nu=1}^{t}a_{\nu}\beta^{1-\nu} + \sum_{\nu=t+1}^{\infty}a_{\nu}\beta^{1-\nu} ) \\
	  &= \sigma \cdot \beta^e (\sum_{\nu=1}^{t}a_{\nu}\beta^{1-\nu} + \beta^{1-t}\sum_{l=1}^{\infty}a_{t+l}\beta^{-l} )
\end{align*}
mit $e_{min} \leq e \leq e_{max}$, dann wird mit $fl(x)$ die gerundete Zahl bezeichnet, wobei $fl(x)$ 
eindeutig gegeben ist durch die Schranke an den \textbf{absoluten Rundungsfehler} \index{Fehler!absoluter Rundungsfehler}
\begin{align*}
	| fl(x) - x | \leq \begin{cases}
								\frac{1}{2}\beta^{e+1+t} & \text{bei symmetrischem Runden}\\
								\beta^{e+1+t}                    & \text{bei Abschneiden}
							\end{cases} \quad .
\end{align*}
Für die \textbf{relative Rechengenauigkeit} \index{Genauigkeit!relative Rechengenauigkeit}folgt somit 
\begin{align*}
\frac{| fl(x) - x | }{|x|} & \leq \begin{cases}
												\frac{1}{2}\beta^{1-t} & \text{bei symmetrischem Runden}\\
												\beta^{1-t}                    & \text{bei Abschneiden}
											\end{cases} \quad .
\end{align*}
Die \textbf{Maschinengenauigkeit} \index{Genauigkeit!Maschinengenauigkeit} des Rechners ist daher durch 
\begin{align*}
	  eps &= \beta^{1-t} & \text{(für float}\approx 10^{-7}  \text{, für double} \approx10^{-16} )
\end{align*}
gegeben.

Die Mantissenlänge bestimmt also die Maschinengenauigkeit. Bei einfacher Genauigkeit ist $fl(x)$ bis auf ungefähr 7 signifikante Stellen genau. \\
Im Folgenden betrachten wir symmetrisches Runden und definieren daher
\[ \tau \coloneqq \frac{1}{2}eps\]
Weiterhin gilt:
\begin{enumerate}[a)]
	\item Die kleinste Zahl am Rechner, welche größer als 1 ist, ist
					\[ 1 + eps \]
	\item Eine Maschinenzahl x repräsentiert eine Eingabemenge
					\[  E(x) = \{\widetilde{x} \in \R : |\widetilde{x}-x| \leq \tau|x|\} \] \\
					IMAGE~MISSING
\end{enumerate}

\subsection{Bemerkung} \label{3.1.7}
Gesetze der arithmetischen Operationen gelten i.A. nicht, z.B.
\begin{itemize}
	\item 	$x$ Maschinenzahl $\quad \Rightarrow fl(x+\nu) = x \text{     für }|\nu| < \tau |x|$
	\item Assoziativ- und Distributivgesetze gelten nicht, z.B. für $\beta = 10, \, t=3, \, a=0,1 ,\, b= 105 , \, c= -104$ gilt:
					\begin{align*}
						fl(a+fl(b+c)) &= 1,1 \\
						fl(fl(a+b)+c) &= fl(fl(105,1) + (-104) ) \\
							                &= fl(105-104) \\
							                &= 1 \quad \lightning
					\end{align*}
	\item[ $\Rightarrow$] Für einen Algorithmus ist die Reihenfolge der Operationen wesentlich!
									  Mathematisch äquivalente Formulierungen können zu verschiedenen Ergebnissen führen.
\end{itemize}

\subsection{Auslöschung von signifikanten Stellen} \label{3.1.8}
Sei $x=9,995\cdot 10^{-1}, y=9,984 \cdot 10^{-1}$. Runde auf drei signifikante Stellen und berechne $x-y$:
\begin{align*}
	\widetilde{f}(x,y) &\coloneqq fl(fl(x)- fl(y)) = fl(1,00\cdot 10^0 - 9,98\cdot 10^{-1}) \\
							  &= 	fl(0,02\cdot 10^{-1}) \\
							  &= fl(2,00 \cdot 10^{-3}) \\
	f(x,y)  &\coloneqq x-y \\
		      &\coloneqq 0,0011 = 1,1\cdot 10^{-3}
	\intertext{Daraus ergibt sich der relative Fehler}
	\frac{|\widetilde{f}(x,y)-f(x,y)|}{|f(x,y)|}
		     &= \frac{|2\cdot 10^{-3}- 1,1\cdot 10^{-3}|}{|1,1\cdot 10^{-3}}
		       = 82\%
\end{align*}
Der Grund hierfür ist, dass das Problem der Substraktion zweier annähernd gleich großer Zahlen
schlecht konditioniert ist.\\

\textbf{Zwei Regeln:}
\begin{enumerate}[1)]
	\item Umgehbare Substraktion annähernd gleich großer Zahlen vermeiden!
	\item Unumgängliche Substraktion möglichst an den Anfang des Algorithmus stellen! (siehe später)
\end{enumerate}

% 2.2
%-----------------------------------------------------------------------------------------------------------------------------------------
\sectione{Kondition eines Problems} \label{3.2}
Es wird das Verhältnis 
\begin{gather*}
	\frac{\text{Ausgabefehler}}{\text{Eingabefehler}}
\end{gather*}
untersucht.

\subsection{Definition: Problem} \label{3.2.1} \index{Problem}
Sei $f: U \subseteq \R^n \rightarrow \R^m$ mit $U$ offen und sei $x\in U$. Dann bezeichne $(f, x)$ das Problem, zu einem gegebenen $x$ die Lösung $f(x)$ zu finden.

\subsection{Definition: absoluter und relativer Fehler} \label{3.2.2} \index{Fehler}
Sei $x\in\R^n$ und $\widetilde{x} \in \R^n$ eine Näherung an $x$. Weiterhin sei $\|\cdot\|$ eine Norm auf $\R^n$.
\begin{itemize}
	\item[a)] $\nn{\widetilde{x} - x}$ heißt \textbf{absoluter Fehler} \index{Fehler!absoluter}
	\item[b)] $\frac{\nn{\widetilde{x} - x}}{\nn{x}}$ heißt \textbf{relativer Fehler}\index{Fehler!relativer}
\end{itemize}
Da der relative Fehler skalierungsinvariant ist, d.h. unabhänging von der  Wahl von $x$ ist, ist dieser i.d.R. von größerem Interesse.
Beide Fehler hängen von der Wahl der Norm ab!
Häufig werden Fehler auch komponentenweise gemessen:
\begin{align*}
	\text{Für } i=1,\cdots , n : && |\widetilde{x}_i - x_i | & \leq \delta & \text{ (absolut)} \\
											 && |\widetilde{x}_i - x_i | &\leq \delta |x_i| & \text{ (relativ)}
\end{align*}

\subsection{Wiederholung: Normen} \label{3.2.3}\index{Norm}

\begin{align*}
	\text{Euklidische Norm ($l_2$-Norm):} &&	\nn{x}_2 &\coloneqq \sqrt{\sum_{i=1}^{n}|x_i|^2}
		\index{Norm!Euklidische Norm} \\
			IMAGE~MISSING \\
	\text{Maximumsnorm ($l_\infty$-Norm):} &&\nn{x}_\infty &\coloneqq \max\{|x_i| : i=1, \cdots n\} \\
	  	\index{Norm!Maximumsnorm}
		  	IMAGE~MISSING \\
	\text{Summennorm ($l_1$-Norm):} &&	\nn{x}_1 &\coloneqq \sum_{i=1}^{n}|x_i| 
		\index{Norm!Summennorm}\\
			IMAGE~MISSING \\
	\text{Hölder-Norm ($l_p$-Norm):} &&	\nn{x}_p &\coloneqq 
		\left(\sum_{i=1}^{n}|x_i|^p\right)^{\frac{1}{p}} 
		\index{Norm!Hölder-Norm}
\end{align*}

\subsection{Definition: Matrixnorm} \label{3.2.4}
Auf dem $\R^n$  sei die Norm $\nn{\,\cdot\,}_a$ und auf dem $\R^m$ die Norm $\nn{\,\cdot\,}_b$ gegeben.
Dann ist die zugehörige \textbf{Matrixnorm} \index{Norm!Matrixnorm} gegeben durch:
\begin{align}
	\nn{A}_{a,b} &\coloneqq \sup_{x\neq 0} \frac{\nn{Ax}_b}{\nn{x}_a} \\ \nonumber
					 &= \sup_{\nn{x}_a=1} \nn{Ax}_b \label{III.2.1} 
\end{align}
Also ist   $\nn{A}_{a,b}$ die kleinste Zahl $c>0$ mit
\begin{gather*}
	\nn{Ax}_b  \leq c\nn{x}_a \quad\quad \forall x\in \R^n
\end{gather*}

\subsection{Definition: Frobeniusnorm, p-Norm, Verträglichkeit} \label{3.2.5}
Sei $A\in \R^{m\times n}$.
\begin{enumerate}[a)]
	\item \textbf{Frobeniusnorm} (Schurnorm):
			 $ \quad \nn{A}_F \coloneqq \sqrt{\sum_{i=1}^{m}\sum_{j=1}^{n}|a_{ij}^2|}
				 \index{Norm!Frobeniusnorm}$
    \item \textbf{p-Norm}: 
			 $\quad \nn{A}_p \coloneqq \nn{A}_{p,p}
				 \index{p-Norm}$
    \item Eine Matrixnorm heißt \textbf{verträglich} \index{Norm!verträglich} mit den Vektornormen 
			    $\nn{\,\cdot\,}_a, \nn{\,\cdot\,}_b$, falls gilt
			    \footnote{ Beachte: $\nn{A}_{a,b}$ ist die kleinste Norm im Gegensatz zu $\nn{A}$, welche hier beliebig ist.}:
				 \begin{gather*}
					 	\nn{Ax}_b \leq \nn{A}\cdot \nn{x}_a \quad \forall x\in \R^n
				 \end{gather*}
\end{enumerate}

\subsection{Bemerkungen} \label{3.2.6}
\begin{enumerate}[a)]
	\item Die Normen $\nn{\,\cdot\,}_F$ und $\nn{\,\cdot\,}_p$ sind \textbf{submultiplikativ} \index{Norm!submultiplikative}, d.h.
				\begin{gather*}
					\nn{A\cdot B} \leq \nn{A}\cdot\nn{B}
				\end{gather*}
	\item Die Norm $\nn{\,\cdot\,}_{1,1}$ wird auch \textbf{Spaltensummennorm}\index{Norm!Spaltensummennorm} genannt:
				\begin{gather*}
					\nn{A}_1 = \max_{1\leq j \leq n}\sum_{i=1}^{m}|a_{ij}|
				\end{gather*}
			Sie ist das Maximum der Spaltensummen\footnote{Beweis: siehe Übungsblatt 3}.
	\item Die Norm $\nn{\,\cdot\,}_{\infty, \infty}$ wird auch \textbf{Zeilensummennorm} \index{Zeilensummennorm}
	 genannt\footnote{Beweis: siehe Übungsblatt 3}:
	%not sure, why \footref won't work here ...
				\begin{gather*}
					\nn{A}_\infty = \max_{1\leq i \leq m}\sum_{j=1}^{n}|a_{ij}|
				\end{gather*}
	\item Die Frobeniusnorm $\nn{\,\cdot\,}_F$ ist verträglich mit der euklidischen Norm $\nn{\,\cdot\,}_2$
	\item Die Wurzeln aus den Eigenwerten von $A^TA$ heißen \textbf{Singulärwerte $\sigma_i$} \index{Singulärwert} von A.
	Mit ihnen kann die $\nn{\,\cdot\,}_{2,2}$ Norm dargestellt werden\footnote{Beweis: siehe Übungsblatt 3}:
				\begin{align*}
					\nn{A}_2 &= max \{\sqrt{\mu} : A^TA\cdot x = \mu x\text{ für ein }x\neq 0 \} \\
								& = \sigma_{max}
				\end{align*}
\end{enumerate}


\marginpar{22.10.2014}
%ultimate evil hack to go along with numeration
%\minisec{\Large3.2 a) Normweise Konditionsanalyse} \label{3.2a}\vspace{1eM}
\extrasection{a)}{Normweise Konditionsanalyse}
%%Alternative to add it to table of contents:
%\renewcommand{\thesubsection}{\thesection.a)}
%\setkomafont{subsection}{\Large}
%\subsection{NORMWEISE KONDITIONSANALYSE}
%\renewcommand{\thesubsection}{\thesection.\arabic{subsection}}
%\addtocounter{subsection}{-1}
%\setkomafont{subsection}{\large}

\subsection{Definition: absolute und relative normweise Kondition}\index{normweise Kondition}
Sei $(f,x)$ ein Problem mit $f:U\subset \R^n \rightarrow \R^m$
und $\nn{\,\cdot\,}_a$ auf $\R^n$ und $\nn{\,\cdot\,}_b$ auf $\R^m$ eine Norm.
\begin{enumerate}[a)]
	\item Die \textbf{absolute normweise Kondition}\index{Kondition!normweise, absolut} eines Problems $(f,x)$ ist die kleinste Zahl 
			 $\kappa _{abs} > 0 $ mit
			 \begin{align}
			 	\nn{f(\widetilde{x})-f(x)}_b &\leq \kappa _{abs}(f,x) \nn{\widetilde{x}-x}_a
			 	+ o\left(\nn{\widetilde{x}-x}_a\right) \label{III.2.2} \\\nonumber
			 	\Bigl(f(\widetilde{x})- f(x) 
						 	&=\underbrace{ f'(x)(\widetilde{x}-x)\pm o\left(\nn{\widetilde{x}-x}\right)}_{Taylorentwicklung}
							 \quad \text{für }\widetilde{x}\rightarrow x 
						 	\Bigr)
			 \end{align}
	\item Die \textbf{relative normweise Kondition}\index{Kondition!normweise. relativ} eines Problems $(f,x)$  mit $x\neq 0, f(x) \neq 0$
			 	ist die kleinste Zahl 
			 	$\kappa _{rel} > 0 $ mit
			 	\begin{align}
			 \frac{	\nn{f(\widetilde{x})-f(x)}_b }{\nn{f(x)}_b}
				 &\leq \kappa _{rel}(f,x)\frac{ \nn{\widetilde{x}-x}_a}{\nn{x}_a}
			 	+ 
			 	o\left(\frac{\nn{\widetilde{x}-x}_a}{\nn{x}_a}\right) \label{III.2.3}
					 &&	\text{für } \widetilde{x} \rightarrow x
			 	\end{align}
	\item Sprechweise:
		\begin{itemize}\index{Kondition!gut/schlecht konditioniert}
			\item falls $\kappa$ \enquote{klein} ist, ist das Problem \enquote{gut konditioniert}
			\item falls $\kappa$ \enquote{groß} ist, ist das Problem \enquote{schlecht konditioniert}
		\end{itemize}
\end{enumerate}

\subsection{Lemma} \label{3.2.8}
Falls $f$ differenzierbar ist, gilt
\begin{gather}
	\kappa_{abs}(f,x) = \nn{Df(x)}_{a,b} \label{III.2.4}
\end{gather}
und für $f(x) \neq 0$
\begin{gather}
	\kappa_{rel}(f,x) = \frac{\nn{x}_a}{\nn{f(x)}_b}\cdot \|Df(x)\|_{a,b} \label{III.2.5}
\end{gather}
wobei $Df(x)$ die Jakobi-Matrix bezeichnet.
%
\subsection{Beispiel: Kondition der Addition}\label{3.2.9} \index{Kondition!Addition}
$f(x_1, x_2) \coloneqq x_1 +x_2 , \, f:\R^2 \rightarrow \R$. \\
Wähle $l_1$-Norm auf $\R^2$ (und $\R$)
\begin{align*}
			Df(x_1, x_2) \, =(\nabla f^T) \, &= (\frac{\partial}{\partial x_1}f, \frac{\partial}{\partial x_2}f )\\
				&= (1,1) && \text{(Matrix!)}
\end{align*}
damit
\begin{align*}
	\kappa_{abs} (f,x)&= \nn{Df(x)}_{1,1} && \text{(Matrix-Norm!!)}\\
						 		&= \nn{Df(x)}_1 \\
								&=1 \\
	\kappa_{rel} (f,x) &= \frac{\nn{x}_1}{\nn{f(x)}_1} \cdot \nn{Df(x)}_{1} \\
								&= \frac{|x_1| + |x_2|}{|x_1+x_2|}
\end{align*}
Daraus folgt: Die Addition zweier Zahlen mit gleichem Vorzeichen ergibt
\begin{gather*}
	\kappa_{rel} = 1
\end{gather*}
Die Subtraktion zweier annähernd gleich großer  Zahlen ergibt eine sehr schlechte relative
Konditionierung:
\begin{gather*}
\kappa_{rel} \gg 1
\end{gather*}
Zum Beispiel in \ref{3.1.8}: Es ist 
\begin{align*}
	x &= \begin{pmatrix}
		9,995 \\
		-9,984
	\end{pmatrix}
	\cdot 10^{-1} \\
	\widetilde{x} = fl(x) &= \begin{pmatrix}
		1 \\
		-9,98\cdot 10^{-1}
	\end{pmatrix}
\intertext{also}
	\frac{|f(\widetilde{x})-f(x)|}{|f(x)|}	&= \frac{0,9}{1,1} 
															= 0,\overline{81} \\
															&\leq \kappa_{rel}(f,x)\cdot \frac{\|\widetilde{x}-x\|_1}{\|x\|_1} \\
															&= \kappa_{rel}(f,x) \cdot 4,6\cdot 10^{-4}
\end{align*}
%

\subsection{Beispiel: Lösen eines Gleichungssystems} \label{3.2.10}
Sei $A\in \R^{n\times n}$ invertierbar und $b\in \R^n$. Es soll 
\begin{gather*}
	Ax =b
\end{gather*}
gelöst werden.
Die möglichen Lösungen in $A$ und in $b$ lassen sich folgendermaßen ermitteln:
\begin{enumerate}[a)]
	\item Betrachte die Störungen in $b$:\\
			Sei hierzu
			\begin{gather*}
			f: b\mapsto x= A^{-1}b 
			\end{gather*}
			Berechne dann $ \kappa(f,b)$ und löse 
			\begin{align*}
					A(x + \Delta x) &= b+\Delta b \\
					f(b + \Delta b) - f(b) &= \Delta x \\
													&= A^{-1} \cdot \Delta b && \text{da }x = A^{-1}b \\
					\Rightarrow \|\Delta x\|_{b}  &= \|A^{-1}\Delta b\|_{b} \\
														&\leq \|A^{-1}\|_{a,b}\cdot \|\Delta b\|_{b} && \forall b, \Delta b 
			\end{align*}
			wobei $\|\cdot\| $ auf $\Renn$ die dem $\Ren$ zugeordnete Matrix-Norm sei. \\
			Die Abschätzung ist \textbf{scharf}\index{scharf}, d.h. es gibt ein $\Delta b\in \R^n$, so dass \enquote{=} gilt, nach Definition \ref{3.2.4}. \\
			Also gilt\footnote{vgl. auch Lemma \ref{3.2.8}: $\kappa_{abs}(f,b)=\nn{Df(b)}_{a,b}=\nn{A^{-1}}_{a,b}$}:
			\begin{gather}
				\kappa_{abs}(f,b) = \nn{A^{-1}}_{a,b} \label{III.2.6}
			\end{gather}
			unabhängig von b.  $ \quad \left( x\mapsto Ax \quad \kappa_{abs}\right)$\\
			Ebenso folgt die scharfe Abschätzung 
			\begin{align}
				\nonumber
				\frac{\|	f(b + \Delta b) - f(b)\|}{\|f(b)\|} &= \frac{\nn{\Delta x}}{\nn{x}}\\ \nonumber
						& = \frac{\nn{A^{-1}\Delta b}}{\nn{x}} \\ \nonumber
						& \leq  \frac{\|A^{-1}\|\cdot \|b\|}{\|x\|} \cdot \frac{\|\Delta b\|}{\|b\|} \\
				\intertext{Damit}
				\kappa_{rel} (f,b) &= \|A^{-1} \| \cdot \frac{\|b\|}{\|A^{-1}\cdot b\|} \label{III.2.7}
			\end{align}
			Da $\|b\| \leq \|A\|\cdot\|x\| = \|A\|\cdot \|A^{-1}b\|$ folgt:
			\begin{gather}
				\kappa_{rel}(f,b) \leq \|A\| \cdot \|A^{-1}\| \label{III.2.8}
			\end{gather}
			für alle (möglichen rechten Seiten) $b $.\\
			\ref{3.2.8} ist scharf in dem Sinne, dass es ein $\widehat{b}\in \R^n$ gibt 
			mit 
			\begin{gather*}
				\|\widehat{b}\| = \nn{A}\cdot \nn{\widehat{x}}
			\end{gather*}
			und somit
			\begin{gather*}
				\kappa_{rel}(f,\widehat{b}) = \nn{A}\cdot \| A^{-1}\|
			\end{gather*} %
			%
	\item Betrachte die Störungen in $A$:\\
			Löse also 
			\begin{gather*}
				(A+\Delta A)(x+\Delta x) = b
			\end{gather*}
			Sei hierzu
			\begin{align*}
				f:A&\mapsto x= A^{-1}b \\
				\R^{n\times n}&\rightarrow \R^n
			\end{align*}
			und berechne $\kappa(f,A)$ mittels Ableitung $Df(A):\R^{n\times n} \rightarrow \R^n$:
			\begin{align*}
				C\mapsto Df(A) C&= \frac{d}{dt} \left((A+tC)^{-1} \cdot b\right) \Big\vert_{t=0} \\
										  & = \frac{d}{dt}\left((A+tC)^{-1}\right)\Big\vert_{t=0}\cdot b
			\end{align*}			
			Weiterhin gilt
			\begin{align}
				\frac{d}{dt} \left((A+tC)^{-1}\right) \Big\vert_{t=0} &= -A^{-1}CA^{-1}, \label{III.2.9}
			\end{align}
			da
			\begin{align*}
				0&= \frac{d}{dt}I \\
				  &= \frac{d}{dt}\left( (A+tC)(A+tC)^{-1}\right)\\
				  &= C(A+tC)^{-1} +(A+tC)\cdot \frac{d}{dt}(A+tC)^{-1} \\
				  \Leftrightarrow \frac{d}{dt} (A+ tC)^{-1} 
				  &= -(A+tC)^{-1} \cdot C(A+tC)^{-1} \, ,
			\end{align*}
			falls $(A+tC)$ invertierbar ist. Für ein genügend kleines $t$ ist das gewährleistet, da $A$ invertierbar ist (s. Lemma \ref{3.2.12}).
			\begin{gather*}
				\Rightarrow Df(A) C = -A^{-1}CA^{-1}b
			\end{gather*}
			Somit folgt
			\begin{align}
				\nonumber
				\kappa_{abs} (f,A) &= \|Df(A)\| \\ \nonumber
											  &= \sup_{\substack{
																C\neq 0 \\ 
																C\in \R^{n\times n}											  	
														  	}}
														  \frac{\|A^{-1}CA^{-1}b\|}{\|C\|} \\ \nonumber
											  &\leq \sup_{\substack{
															 	C\neq 0 \\ 
															 	C \in \R^{n\times n}														  	
															 }}
												 \frac{\|A^{-1}\|\cdot\|C\|\cdot\|A^{-1}b\|}{\|C\|} \\ \nonumber
											  &= \|A^{-1}\| \cdot\|x\| \\ \nonumber
											  &\leq   \|A^{-1}\|^2 \cdot\|b\| \\ \nonumber
				 \kappa_{rel}(f,A)  &= \frac{\|A\|}{\|f(A)\|} \cdot \|Df(A)\| \\
											 &\leq \|A\|\cdot \|A^{-1}\| \label{III.2.10}
			\end{align}
	 \item betrachte Störungen in $A$ und $b$ :
		 \begin{gather*}
		 	(A+\Delta A)(x+\Delta x) = (b+\Delta b) 
		 \end{gather*}
		 Für $\kappa$ müsste $\|(A,b)\|$ festgelegt werden. Dies wird jedoch nicht betrachtet. Es gilt aber folgende Abschätzung für invertierbare Matrizen $A\in \Renn $ und Störungen
		 $\Delta A \in \R^{n\times n}$ mit $\|A^{-1}\|\cdot \|\Delta A\| < 1$:
		 \begin{align}
			 \frac{\|\Delta x\|}{\|x\|} & \leq \|A\| \cdot \|A^{-1}\|\cdot (1- \|A^{-1}\|\cdot \|\Delta A\|) 
													 \cdot
													 \underbrace{\left(  \frac{\|\Delta b\|}{\|b\|} +  \frac{\|\Delta A\|}{\|A\|}  \right)}_{\neq  \frac{\|(\Delta A, \Delta b)\|}{\|(A,b)\|} }
													 \label{III.2.11}
		 \end{align}
		 \paragraph{Beweis:} s. Übungsblatt
\end{enumerate}

\subsection{Definition: Kondition einer Matrix} \index{Kondition!Matrix}
Sei $\|\cdot\|$ eine Norm auf $\R^{n\times n} $ und $A\in \R^{n\times n}$ eine reguläre Matrix.
Die Größe
\begin{gather*}
	\kappa_{\|\cdot\|}(A) = cond_{\|\cdot\|} \coloneqq \|A\| \cdot \|A^{-1}\|
\end{gather*}
heißt \textbf{Kondition der Matrix} bzgl. der Norm ${\|\cdot\|}$. \\
Ist  ${\|\cdot\|}$ von einer Vektor-Norm ${\|\cdot\|}_p$ induziert, bezeichnet 
	$cond_p(A)$
die $cond_{\|\cdot\|_p}(A)$. Wir schreiben $cond(A)$ für $cond_2(A)$. \\
$cond_{\|\cdot\|}(A) $ schätzt die relative Kondition eines linearen GLS $Ax=b$ für alle möglichen 
Störungen in $b$ oder in $A$ ab und diese Abschätzung ist scharf. \\

Es stellt sich nun die Frage: \\
\textit{Wann existiert die Inverse der gestörten invertierbaren Matrix $A$?} \\
Hierzu werden wir die Relationen benötigen:
\begin{align*}
	A+\Delta A &= A (I+A^{-1}\Delta A)\\
\intertext{und mit $C \in \Renn,\, \|C\| < 1$}
	(I-C)^{-1} &= \sum_{k=0}^{\infty}C^k \\
	\|	(I-C)^{-1} \| &\leq \frac{1}{1-\|C\|}
\end{align*}

\marginpar{27.10.2014}
\subsection{Lemma (Neumannsche Reihe)}\index{Neumannsche Reihe}\label{3.2.12}
\addtocounter{equation}{1}
Sei $C\in\Renn$ mit $\|C\|<1$ und mit einer submultiplikativen Norm $\|\cdot\|$,
so ist $(I-C)$ invertierbar und es gilt:
\begin{gather*}
	(I-C)^{-1}=\sum_{k=0}^{\infty}C^k
\end{gather*}
Weiterhin gilt:
\begin{gather*}
	\|(I-C)^ {-1}\| \leq \frac{1}{1-\|C\|}
\end{gather*}

\paragraph{Beweis}
Es gilt zu zeigen, dass $\sum_{k=1}^{\infty}C^k$ existiert: \\
Sei $q\coloneqq \|C\| < 1$, dann gilt: 
\begin{align*}
	\nn{ \sum_{k=0}^{m} C^k } &\leq \sum_{k=0}^{m} \nn{C^k }  && \text{Dreiecksungleichung} \\
								 &\leq \sum_{k=0}^{m}\nn{C}^k && \text{da $\nn{\,\cdot\,}$ submultiplikativ}\\
								 &=\sum_{k=0}^{m}q^k  \\
								 &= \frac{1-q^{m+1}}{1-q} \\
								 &\leq \frac{1}{1-\nn{C}} && \forall m\in \N, \text{ da } q<1 \text{ (geometr. Reihe)}
\end{align*}
Daraus folgt bereits, dass $\sum_{k=1}^{\infty}C^k$ existiert (nach Majorantenkriterium).\\
Weiter gilt dann:
\begin{align*}
	(I-C) \sum_{k=1}^{\infty}C^k &= \lim\limits_{m\rightarrow \infty}(I-C)   \sum_{k=1}^{m}C^k \\
													&= \lim\limits_{m\rightarrow \infty} (C^0-C^{m+1}) \\
													&=I  &&\square
\end{align*}
\subsection{Bemerkung}
\begin{enumerate}[a)]
	\item Für symmetrische, positiv definite Matrix $A\in \Renn$ gilt\footnote{Beweis: siehe Übungsblatt 3}: 
	\begin{gather}
		\kappa_2(A) = \frac{\lambda_{max}}{\lambda_{min}} \label{III.2.13}
	\end{gather}
	\item Eine andere Darstellung von $\kappa(A)$ ist
	\begin{gather}
		\kappa(A) \coloneqq 
		\frac{\underset{\|x\|=1}{\max}\|Ax\|}{\underset{\|x\|=1}{\min}\|Ax\|} \in  \left[ 0, \infty \right]
			 \label{III.2.14}
	\end{gather}
	Diese ist auch für nicht invertierbare und rechteckige Matrizen wohldefiniert. \\
	Dann gilt offensichtlich:
	\item $\kappa(A) \geq 1$
	\item $\kappa(\alpha A)=\kappa(A) \quad \text{für } 0\neq\alpha\in\R$ (skalierungsinvariant)
	\item $A\neq 0$ und $A\in\Renn $ ist genau dann singulär, wenn $\kappa(A)=\infty$. \\
			Wegen der Skalierungsinvarianz ist die Kondition zur Überprüfung der Regularität von $A$ 
			besser geeignet als die Determinante.
	\end{enumerate}
	

\subsection{Beispiel: Kondition eines nichtlinearen Gleichungssystems}
Sei $f:\Ren\rightarrow\Ren$ stetig differenzierbar und $y\in\Ren$ gegeben. \\
Löse
\begin{gather*}
	f(x) = y
\end{gather*}
Gesucht:
\begin{gather*}
	\kappa(f^{-1},y)
\end{gather*}
mit $f^{-1}$ Ausgabe und $y$ Eingabe. \\
Sei $Df(x)$ invertierbar, dann existiert aufgrund des Satzes für implizite Funktionen die inverse Funktion $f^{-1}$ lokal in einer Umgebung von $y$ mit $f^{-1}(y)=x$, sowie
\begin{gather*}
	D(f^{-1})(y) = (Df(x))^{-1}
\end{gather*}
Hiermit folgt:
\begin{align}
	\nonumber
	\kappa_{abs}(f^{-1},y) &= \|(Df(x))^{-1}\| \\
	\kappa_{rel}(f^{-1},y) &= \frac{\|f(x)\|}{\|x\|}\cdot\|(Df(x))^{-1}\|  \label{III.2.15}
\end{align}
Für skalare Funktionen $f:\R\rightarrow\R$ folgt somit:
\begin{gather*}
	\kappa_{rel}(f^{-1},y) = \frac{|f(x)|}{|x|}\cdot \frac{1}{|f'(x)|}
\end{gather*}
Falls $|f'(x)|\rightarrow 0$ ist es eine schlechte absolute Kondition. \\
Für $|f'(x)| \gg 0$ ist es eine gute absolute Kondition.\\

IMAGE~MISSING\\

Damit bedeutet eine kleine Störung in $y$ eine große Störung in $x$.


%\minisec{\Large3.2 b) Komponentenweise Konditionsanalyse} \label{3.2b}\vspace{1eM}
\extrasection{b)}{Komponentenweise Konditionsanalyse}

\subsection{Beispiel}
Falls $A$ Diagonalgestalt hat, sind die Gleichungen unabhängig voneinander (entkoppelt)\index{entkoppelt}.
Die erwartete relative Kondition wäre dann -- wie bei skalaren Gleicungen -- stets gleich 1.
Ebenso sind Störungen nur in der Diagonale zu erwarten. Jedoch:
\begin{align*}
	A  &=\begin{pmatrix}
					1 & 0\\
					0 & \epsilon
				\end{pmatrix} \\
	\Rightarrow 	A^{-1}&=\begin{pmatrix}
													1 & 0\\
													0 & \epsilon^{-1}
												\end{pmatrix}\\
	\Rightarrow \kappa_\infty& = \kappa_2 = \frac{1}{\epsilon} 
													&& \text{für }0 < \epsilon \leq 1											
\end{align*}

\subsection{Definition: Komponentenweise Kondition}\index{Kondition!komponentenweise}
Sei $(f, x) $ ein Problem mit $f(x)\neq 0$ und $x=(x_i)_{i=1,\cdots , n}$ mit $x_i\neq 0 $  für alle $i=1,\cdots, n$.
Die \textbf{komponentenweise Kondition} von $(f,x) $ ist die kleinste Zahl $\kappa_{rel}\geq 0$, so dass:
\begin{align*}
	\frac{\|f(\widetilde{x})-f(x)\|_\infty}{\|f(x)\|_\infty} 
				&\leq \kappa_{rel} \cdot \underset{i}{\max}\frac{|\widetilde{x_i}-x_i|}{|x_i|}+ o\left(\underset{i}{\max}\frac{|\widetilde{x_i}-x_i|}{|x_i|}\right) 
				&& \text{für }\widetilde{x}\rightarrow x
\end{align*}
Vorsicht:
\begin{gather*}
	\frac{\|\widetilde{x}-x\|_\infty}{\|x\|_\infty}\neq \underset{i}{\max}\frac{|\widetilde{x_i}-x_i|}{|x_i|}
\end{gather*}

\subsection{Lemma} \label{3.2.17}
Sei $f$ differenzierbar und fasse $|\cdot|$ komponentenweise auf, d.h. $|x| = \begin{pmatrix}
																																						|x_1| \\
																																						\vdots \\
																																						|x_n|
																																					\end{pmatrix}$.
Dann gilt:
\begin{gather}
	\kappa_{rel} = \frac{\|\, |Df(x)|\cdot |x| \, \|_\infty}{\|f(x)\|_\infty} \label{III.2.16}
\end{gather}

\paragraph{Beweis}
Vergleiche seien ebenfalls komponentenweise zu verstehen. \\
Nach dem Satz von Taylor gilt: 
\begin{align*}
	f_i(\widetilde{x})-f_i(x) 
						&= \left( \frac{\partial f_i}{\partial x_i}(x), \cdots ,\frac{\partial f_i}{\partial x_n}(x) \right)
								\cdot \begin{pmatrix}
									\widetilde{x}_1-x_1 \\
									\vdots \\
									\widetilde{x}_n-x_n
								\end{pmatrix}
							+ o\left(\|\widetilde{x}-x\|\right) \\
	\Rightarrow |f_i(\widetilde{x})-f_i(x)|
						&\leq |Df(x)|
							\cdot \begin{pmatrix}
									|x_1|\cdot \frac{\widetilde{x}_1-x_1 }{|x_1|}\\
									\vdots \\
									|x_n|\cdot \frac{\widetilde{x}_n-x_n }{|x_n|}
								\end{pmatrix}
									+ o\left(\underset{i}{\max}\frac{\widetilde{x}_i-x_i }{|x_i|}\right) 
									&& \text{da $x_i$ fest und $\widetilde{x}_i\rightarrow x_i$} \\
	&\leq |Df(x)| \cdot |x| \cdot \underset{i}{\max}\frac{\widetilde{x}_i-x_i }{|x_i|}
									+o\left(\underset{i}{\max}\frac{\widetilde{x}_i-x_i }{|x_i|}\right) \\
	\Rightarrow \frac{\|f(\widetilde{x})-f(x)\|_\infty}{\|f(x)\|_\infty}
						&\leq  \frac{\|\, \|Df(x)|\cdot |x| \, \|_\infty}{\|f(x)\|_\infty}
								\cdot \underset{i}{\max}\frac{\widetilde{x}_i-x_i }{|x_i|}
								+ o\left( \underset{i}{\max}\frac{\widetilde{x}_i-x_i }{|x_i|} \right)
\end{align*}
Wähle $\widetilde{x}_i = x_j+h\cdot sign \frac{\partial f_i}{\partial x_j}(x)$ mit $h>0$,
dann gilt:
\begin{gather*}
	|Df_i(x)(\widetilde{x}-x)| = Df_i(x)(\widetilde{x}-x)
\end{gather*}
und in obiger Rechnung gilt Gleichheit. \\
Also folgt, dass
\begin{align*}
	 \frac{\|\,|Df(x)|\cdot |x| \, \|_\infty}{\|f(x)\|_\infty} &= \kappa_{rel}  \\
	 &&& \square
\end{align*}

\subsection{Beispiel}
\begin{enumerate}[a)]
	\item Komponentenweise Kondition der Multiplikation
				\begin{align*}
					f:&\R^2 \rightarrow \R, \, f(x,y) \coloneqq x\cdot y \\
					   \Rightarrow Df(x,y) &= (y, x)  \\
					   \Rightarrow \kappa_{rel}(x,y) &= \frac{\left\| (|y|, |x|)\cdot \begin{pmatrix}
																																   	|x| \\
																																   	|y|
																															   	\end{pmatrix}\right\|_\infty}
										{|x\cdot y|} \\
						&= \frac{2\cdot|x|\cdot |y|}{|x\cdot y|} \\
						&= 2
				\end{align*}
	\item Komponentenweise Kondition eines linearen Gleichungssystems:\\
				Löse $Ax=b$ mit möglichen Störungen in $b$, also zu
				\begin{align*}
					f: & b\mapsto A^{-1}b \\
					\kappa_{rel} & = \frac{\| \, |A^{-1}| \cdot |b|\, \|_\infty}{\|A^{-1}b\|_\infty}
				\end{align*}
				Falls A eine Diagonalmatrix ist, folgt:
				\begin{gather*}
					\kappa_{rel}=1
				\end{gather*}
	\item Komponentenweise Kondition des Skalarproduktes:
				\begin{align*}
					\langle x,y \rangle \coloneqq \sum_{i=1}^{n}x_i y_i& = x^Ty \\
					f: \R^2 \rightarrow \R, \, f(x,y) &= \langle x,y \rangle \\
					\Rightarrow Df(x,y) &= (y^T, x^T) \\
					\kappa_{rel}  &= \frac{\left\| \,\left|(y^T, x^T)\right|\cdot\left|\begin{pmatrix}
																											x \\
																											y
																								 	\end{pmatrix}\right|\right\|_\infty }
																{\|\langle x,y\rangle\|_\infty}\\
												&= \frac{2\cdot |y^T|\cdot |x|}{|\langle x,y\rangle|} \\
												&= 2\cdot \frac{\langle |x|,|y|\rangle}{|\langle x,y\rangle|} \\
												&= 2 \cdot \frac{\cos(|x|, |y|)}{\cos(x,y)}  \\
	&&&				\text{	da  }\cos(x,y) = \frac{\langle y,x \rangle}{\|x\|_2 \cdot \|y\|_2} \, . 
				\end{align*}
				Falls $x$ und $y$ nahezu senkrecht aufeinander stehen, kann das Skalarprodukt sehr schlecht konditioniert sein. \\
				Zum Beispiel für $x=\widetilde{x} = \begin{pmatrix} 1 \\1 \end{pmatrix}$
					 und $y=\begin{pmatrix} 1+10^{-10} \\-1 \end{pmatrix},
					  \, \widetilde{y}=\begin{pmatrix} 1 \\-1 \end{pmatrix}$. \\
					  IMAGE~MISSING
\end{enumerate}

\sectione{Stabilität von Algorithmen}
Bislang: Kondition eines gegebenen Problems $(f,x)$. \\
Nun stellt sich die Frage: \textit{Was passiert durch das Implementieren am Rechner? }\\
Ein \enquote{stabiler} Algorithmus sollte ein gut konditioniertes Problem nicht \enquote{kaputt machen}.\\

IMAGE~MISSING

%\minisec{\Large3.3 a) Vorwärtsanalyse} \label{3.3a}\vspace{1eM}
\extrasection{a)}{Vorwärtsanalyse}

Die Fehlerfortpflanzung durch die einzelnen Rechenschritte, aus denen die Implementierung aufgebaut ist, wird abgeschätzt.

\subsection{Bemerkung}
Für die Rechenoperationene $+,-,\, \cdot \, , \, /\,$, kurz $\nabla$, gilt:
\begin{align}
	\nonumber
	fl(a\nabla b) &= (a\nabla b)\cdot (1+\epsilon) \\
						   &= (a\nabla b) \cdot \frac{1}{1+\mu} \label{III.3.1}
\end{align}
mit $|\epsilon|, |\mu| \leq eps$.


\marginpar{29.10.2014}
\subsection{Beispiel}
Sei $f(x_1, x_2, x_3) \coloneqq \frac{x_1x_2}{x_3}$ mit Maschinenzahlen $x_i$ und $x_3\neq 0$ und sei der Algorithmus durch
\begin{gather*}
	f(x_1, x_2, x_3) = (f^{(2)} \circ f^{(1)})(x_1, x_2, x_3) 
\end{gather*}
gegeben mit 
\begin{align*}
	f^{(1)}(x_1, x_2, x_3) & = (x_1\cdot x_2, x_3) && \text{und} \\
	f^{(2)}(y,z) &= \frac{y}{z}
\end{align*}
Die Implementierung $\widetilde{f}$ von $f$  beinhaltet Rundungsfehler. \\

Sei  $x=(x_1, x_2, x_3) $. Daraus folgt:
\begin{align*}
	\widetilde{f}^{(1)}(x) &= (fl(x_1\cdot x_2), x_3) \\
										& = (x_1x_2 (1+\epsilon_1), x_3)
\intertext{mit $|\epsilon_1|\leq eps$:}
	\widetilde{f}(x) &= \widetilde{f}^{(2)}(\widetilde{f}^{(1)}(x)) \\
							&= fl(f^{(2)}(x_1 x_2 (1+\epsilon_1), x_3)) \\
							&= \frac{x_1x_2(1+\epsilon_1)}{x_3}\cdot (1+\epsilon_2)  \\
							&= f(x)\cdot (1+\epsilon_1)(1+\epsilon_2)
\intertext{mit $|\epsilon_2| \leq eps$:}
	\frac{|\widetilde{f}(x) -f(x)|}{|f(x)|} &= |\epsilon_1+\epsilon_2 +\epsilon_1\cdot \epsilon_2| \\
								&\leq 2eps + eps^2
\end{align*}
Dies ist eine \enquote{worst case} Analyse, da immer der maximale Fehler angenommen wird,
und gibt i.d.R. eine starte Überschätzung des Fehlers an.
Für qualitative Aussagen sind sie jedoch unnützlich. \\
In Computersystemen stehen mehr Operationen wie $\nabla$ zur Verfügung,
die mit einer relativen Genauigkeit $eps$ realisiert werden können. \\

Daher:

\subsection{Definition: Elementar ausführbar}
Eine Abbildung $\phi : U\subseteq \Ren \rightarrow \R^m$ heißt
\textbf{elementar ausführbar}\index{elementar ausführbar}, falls es 
eine elementare Operation $\widetilde{\phi}:\F^n \rightarrow \F^m$
gibt, wobei $\F$ die Menge der Maschinenzahlen bezeichne mit
\begin{gather}
	|\widetilde{\phi}_i(x)-\phi_i(x)| \leq eps\cdot |\phi_i(x) | 
	\quad \forall x\in \F^n \text{ und } i=1,\cdots , m \label{III.3.2}\, .
\end{gather}
$\widetilde{\phi}$ heißt dann \textbf{Realisierung}\index{Realisierung} von $\phi$.

\paragraph{Bemerkung:}
aus \eqref{III.3.2} folgt für $1\leq p\leq \infty$:
\begin{gather}
	\nn{\widetilde{\phi}(x)-\phi(x)}_p \leq eps\cdot\nn{\phi(x)}_n 
	\quad \forall x\in\F^n \label{III.3.3}
\end{gather}


\subsection{Definition: Algorithmus, Implementation}
Sei $f:E\subseteq \Ren \rightarrow \R^m$ gegeben.\\
Ein Tupel $\left(f^{(1)},\cdots ,f^{(l)}\right)$ mit $l\in \N$ von elementar ausführbaren
Abbildungen
\begin{gather*}
	f^{(i)}: U_1\subseteq \R^{k_i} \rightarrow U_{i+1}\subseteq \R^{k_{i+1}}
\end{gather*}
mit $k_1=n$ und $k_{l+1}=m$ heißt \textbf{Algorithmus}\index{Algorithmus} von $f$, falls
\begin{gather*}
	f=f	^{(l)}\circ \dotsc \circ f^{(1)}
\end{gather*}
Das Tupel $(\widetilde{f}1^{(1)},\cdots ,\widetilde{f}^{(l)})$ mit Abbildungen $\widetilde{f}^{(i)}$, welche Realisierungen der $f^{(i)}$ sind,
heißt \textbf{Implementation}\index{Implementation} von 
$\left(f^{(1)},\dotsc ,f^{(l)}\right)$.
Die Komposition 
\begin{gather*}
		\widetilde{f}=\widetilde{f}	^{(l)}\circ \dotsc \circ \widetilde{f}^{(1)}
\end{gather*}
heißt Implementation von f. \\
Im Allgemeinen gibt es verschiedene Implementierungen einer Abbildung $f$.

\subsection{Lemma (Fehlerfortpflanzung)}\label{3.3.5} \index{Fehler!Fortpflanzung}
Sei $x\in \Ren$ und $\widetilde{x}\in \F^n$ mit $|\widetilde{x}_i-x_i|\leq eps|x_i|$ für alle 
$i=1,\cdots , n$.
Sei $\left(f^{(1)},\dotsc ,f^{(l)}\right)$ ein Algorithmus für $f$ und 
$(\widetilde{f}^{(1)},\dotsc ,\widetilde{f}^{(l)})$ eine zugehörige Implementation. \\
Mit den Abkürzungen
\begin{align*}
	x^{(j+1)} &\coloneqq f^{(j)}\circ \dotsc \circ f^{(1)}(x) \\
	x^{(1)} &\coloneqq x
\end{align*}
und entsprechend mit $\widetilde{x}^{(j+1)}$ gilt,
falls $x^{(j+1)} \neq 0$ für alle $j=0,\dotsc , (l-1)$ und $\nn{\,\cdot\,}$ eine beliebige p-Norm ist:
\begin{align}
	\frac{\nn{\widetilde{x}^{(j+1)}-x^{(j+1)}}}{\nn{x^{(j+1)}}}
	&\leq eps \cdot \K + o\left(eps\right)
	\label{III.3.4} 
	\\ \nonumber
	\K^{(j)}&=(1+\kappa^{(j)}+\kappa^{(j)}\cdot \kappa^{(j-1)}+ \cdots + \kappa^{(j)}\cdot \dotsm \cdot \kappa^{(1)}) \\ \nonumber
\end{align}
wobei $	\kappa^{(j)} \coloneqq \kappa_{rel}(f^{(j)}, x^{(j)})$ die Kondition der elementar ausführbaren Operationen $f^{(j)}$ ist.

\paragraph{Beweis}

\begin{align*}
	\frac{\nn{\widetilde{x}^{(j+1)}-x^{(j+1)}}}{\nn{x^{(j+1)}}}
				&= \frac{\nn{\widetilde{f}^{(j)}(\widetilde{x}^{(j)})-f^{(j)}(x^{(j)})}}
				{\nn{f^{(j)}(x^{(j)})}} \\
				&\leq \frac{\nn{\widetilde{f}(\widetilde{x})-f(\widetilde{x})}}{\nn{f(\widetilde{x})}}
						\cdot \frac{\nn{f(\widetilde{x})}}{\nn{f(x)}}
						+ \frac{\nn{f(\widetilde{x})-f({x})}}{\nn{f(x)}} 
						\quad\quad \text{(Index $j$ vernachlässigt)}\\
				&\leq eps \left( 1+ \frac{\nn{f(\widetilde{x})-f({x})}}{\nn{f(x)}}\right)
						+ \frac{\nn{f(\widetilde{x})-f({x}})}{\nn{f(x)}}\\
				&\overset{\text{nach \ref{III.3.3}}}{=} eps + (eps+1) \cdot 						\left(\kappa{(j)}\cdot \frac{\nn{\widetilde{x}^{(j)}-x^{(j)}}}{\nn{x^{(j)}}}\right)
						+ o\left( \frac{\nn{\widetilde{x}^{(j)}-x^{(j)}}}{\nn{x^{(j)}}}\right)
\end{align*}
Nach Voraussetzung gilt Gleichung \eqref{III.3.4}  mit $\K^{(0)}=1$ für $j=0$. \\
Für $j=1$ folgt nach Voraussetzung mit Gleichung \eqref{III.3.3}
\begin{align*}
	\frac{\nn{\widetilde{x}^{(2)}-x^{(2)}}}{\nn{x^{(2)}}}
	 & \leq eps +(eps+1) \cdot \left( \kappa^{(1)}eps+ o(eps)\right) \\
	 &= eps(1+\kappa^{(1)}) + o(eps) \\
	 &= eps\K^{(1)} + o(eps)
\end{align*}
Womit der Induktionsanfang gezeigt ist. \\
Für den Induktionsschritt von $j-1$ zu $j$:
\begin{align*}
	\frac{\nn{\widetilde{x}^{(j+1)}-x^{(j+1)}}}{x^{(j+1)}}
		& \leq eps + (1+eps)\kappa^{(j)} \left[ eps \K^{(j-1)}+ o(eps) \right] \\
		&\phantom{\leq eps+} + (1+eps) \cdot o\left( eps\cdot \K^{(j-1)} +o(eps)\right) \\
		&= eps\left(1+\kappa^{(j)}\cdot \K^{(j-1)}\right)+ o(eps)
\end{align*}
Mit $\K^{(j)} = 1+ \kappa^{(j)}\cdot \K^{(j-1)}$ folgt die Behauptung.
\hfill $\square$
\\

Hiermit folgt:

\subsection{Korollar}\label{3.3.6}
Unter der Voraussetzung von Lemma \ref{3.3.5} gilt:
\begin{gather}
	\frac{\nn{\widetilde{f}(\widetilde{x})-f(x)}}{\nn{f(x)}} \leq 
		eps\cdot \left( 1+\kappa^{(l)}+ \kappa^{(l)}\cdot \kappa^{(l-1)}+ \dotsc
			+ \kappa^{(l)}\cdot \dotsc \cdot \kappa^{(1)}\right) + o(eps) 
			\label{III.3.5}
\end{gather}

\subsection{Bemerkung}
Mit Korollar \ref{3.3.6} ist offensichtlich, dass schlecht konditionierte Probleme 
zu elementar ausführbaren Abbildungen so früh wie möglich ausgeführt werden sollten. \\
Nach Beispiel \ref{3.2.9} ist die Substraktion zweier annähernd gleicher Zahlen schlecht konditioniert.
Deshalb sollte man unvermeidbare Subtraktionen möglichst früh durchführen. \\
Allerdings hängt $\kappa^{(j)}$ nicht nur von $f^{(j)}$, sondern auch vo\nocite{*}m Zwischenergebnis $x^{(j)}$ ab, welches a priori unbekannt ist.

\subsection{Bemerkung zur Sprechweise} %\label{3.3.8}
Der Quotient 
\begin{gather}
	\frac{\overbrace{\frac{\nn{\widetilde{f}(\widetilde{x})-f(x)}}{\nn{f(x)}}}^{
				\text{Gesamtfehler}}}
		{\underbrace{\frac{\nn{\widetilde{f}(\widetilde{x})}}{\nn{f(x)}}}_{
				\scriptsize\substack{
					\text{Fehler} \\
					\text{durch} \\
					\text{Problem}
				}}
			\cdot
		 {\underbrace{\frac{\nn{\widetilde{x}-x}}{\nn{x}}}_{
		 		\scriptsize\substack{\text{Eingabe-} \\ \text{fehler}}}}}
	\label{III.3.6}
\end{gather}
gibt die \textbf{Güte des Algorithmus} \index{Güte!Algorithmus} an.
Als Stabilitätsindikator kann also 
\begin{gather}
	\sigma\left(f, \widetilde{f}, x\right) \coloneqq \frac{\K}{\kappa_{rel}(f, x)}
	\label{III.3.7}
\end{gather}
 verwendet werden und es gilt
 \begin{gather*}
	\frac{\nn{\widetilde{f}(\widetilde{x})-f(x)}}{\nn{f(x)}}
		< \underbrace{\sigma\left( f,\widetilde{f}, x\right) }_{
								\substack{\text{Beitrag}\\
												 \text{des} \\
												 \text{Algorithmus}}}
		   \cdot \underbrace{\kappa_{rel}(f,x)}_{
		   						\substack{\text{Beitrag} \\
		   										 \text{des} \\
		   										 \text{Problems}}}
		   	\cdot \underbrace{eps}_{\substack{\text{Rundungs-}\\\text{fehler}}}
		    + \quad o(eps)
 \end{gather*}
Falls $\sigma( f,\widetilde{f}, x)  < 1$, dämpft der Algorithmus die Fehlerfortpflanzung der Eingabe- und Rundungsfehler und heißt \textbf{stabil}\index{Stabilität}. \\
Für $\sigma( f,\widetilde{f}, x)  \gg 1$ heißt der Algorithmus \textbf{instabil}.



\subsection{Beispiel}
Nach Gleichung \eqref{III.3.3} gilt für die Elementaroperationen $\K\leq 1$.
Da für die Subtraktion zweier annähernd gleich großer Zahlen $\kappa_{rel}\gg 1$ gilt,
ist der Stabilitätsfaktor zweier annähernd gleich großer
Zahlen sehr klein und der Algorithmus also stabil, Falls es sich jedoch bei einer zusammengesetzten Abbildung $f=h\circ g$  bei der zweiten Abbildung $h$ um eine Subtaktion handelt, gilt
\begin{gather*}
	\K =(1+\kappa(sub)+\kappa(sub)\cdot\kappa(g))
\end{gather*}
und die Stabilität ist gefährdet.
Genauere Abschätzungen und damit genauere Indikatoren können durch komponentenweise Betrachtungen erhalten werden.


\extrasection{b)}{Rückwärtsanalyse} \index{Rückwärtsanalyse}
Die Fragestellung ist nun: \\
\textit{Kann $\widetilde{f}(\widehat{x})$ als exaktes Ergebnis von einer gestörten Eingabe $\widehat{x}$ unter der exakten Abbildung $f$ aufgefasst werden?}\\
Das würde heißen
\begin{gather*}
	\exists\, \widehat{x}\in \Ren: f(\widehat{x})= \widetilde{f}(\widetilde{x}) \, .
\end{gather*}
Dann schätze den Fehler 
\begin{gather*} 
	\nn{\widehat{x}-x}
\end{gather*}
 bzw. für nicht injektive $f$
 \begin{gather*}
	\min_{\widehat{x}\in \Ren}
\left\{
 	\nn{\widehat{x}-x} 
 					\middle\vert f(\widehat{x}) = \widetilde{f}(\widetilde{x}) 
 					\right\}
 \end{gather*} 
  ab. \\
  
  IMAGE~MISSING
  
  
  \subsection{siehe Folien}
  (siehe auch Folien) \\
  
  IMAGE~MISSING
  
  \sectione{Beurteilung von Näherungslösungen linearer GLS}
  Zu $Ax=b$ liege eine Näherungslösung $\widetilde{x}$ vor.

  \begin{enumerate}[a)]
  	\item Im Sinne der Vorwärtsanalyse und der Fehlerentwicklung	
  			durch das Problem gilt:
  				\begin{gather*}
  					\frac{\nn{\widetilde{x}-x}}{\nn{x}} \leq cond(A) \cdot \frac{\nn{\Delta b}}{\nn{b}}
  				\end{gather*}
  			nach Beispiel \ref{3.2.10}, 
  			mit dem Residuum 
  			\begin{align}
  				r(\widetilde{x})  & \coloneqq A\widetilde{x} - b \label{III.4.1} \\ \nonumber 
  										  &	= \widetilde{b}-b \\ \nonumber
  										  & = \Delta b
  			\end{align}
  			Wie der absolute Fehler ist das Residuum skalierungsabhängig.
  			Daher ist $\nn{r(\widetilde{x})}$ \enquote{klein} ungeeignet, um
  			Genauigkeitsaussagen zu treffen. \\
  			Um den Fehler in $x$ abzuschätzen, ist die Betrachtung von 
  			\begin{gather}
  				\frac{\nn{r(\widetilde{x})}}{\nn{b}} \label{III.4.2}
  			\end{gather}
  			geeigneter. \\
  			Für große $cond(A)$ ist dieser Quotient jedoch weiterhin ungeeignet.
  			
  			\item siehe Folien
  \end{enumerate}
  
  
  \chapter{Lineare Gleichungssysteme: Direkte Methoden (Fortsetzung)}
  
  \sectione{Gaußsches Eliminationsverfahren mit Aquilibrierung und Nachiteration}
  
  Mit Skalierung $D_zA$ (\textbf{Zeilenskalierung})\index{Skalierung!Zeilen-} oder
  $D_sA$ (\textbf{Spaltenskalierung})\index{Skalierung!Spalten-}
  mittels Diagonalmatrizen $D_z, D_s$ lässt sich eine Pivotstrategie beliebig abändern.
  Jetzt ist die Frage: \\
  \textit{Was ist eine \enquote{gute} Skalierung?}
  
  Skalierung ändert die Lönge der Basisvektoren des Bild- bzw. des Urbildvektorraumes.
  Durch Normierung der Länge auf 1 wird die Pivotstrategie unabhängig von der 
  gewählten Einheit.
  
    Sei $A\in\Renm $ und $\nn{\,\cdot\,} $ eine Vektornorm.
  
  
  \subsection{Äquilibrierung der Zeilen} \index{Äquilibrierung!Zeilen-}
  Alle Zeilen von $D_zA$ haben die gleiche Norm, z.B. $\nn{\,\cdot\,} =1$, wofür 
  \begin{gather}
  	D_z = \begin{pmatrix}
  	\sigma_1 & & 0 \\
  	&\ddots & \\ 
  	0 && \sigma_n
  	\end{pmatrix}
  	 \quad \text{ mit }\sigma_i\coloneqq \frac{1}{\nn{(a_{i1}, \dots , a_{im})}}
  	 \label{IV.1.1}
  \end{gather}
  gesetzt wird.


  \subsection{Äquilibrierung der Spalten} \index{Äquilibrierung!Spalten-}
  Alle Spalten von $AD_s$ haben die gleiche Norm, z.B. $\nn{\,\cdot\,} =1$, wofür 
  \begin{gather}
  D_s = \begin{pmatrix}
  \tau_1 & & 0 \\
  &\ddots & \\ 
  0 && \tau_m
  \end{pmatrix}
  \quad \text{ mit }\tau_j\coloneqq \nn{\begin{pmatrix}
  		a_{1j} \\ \vdots \\ a_{nj}
  		\end{pmatrix}}^{-1}
  \label{IV.1.2}
  \end{gather}
  gesetzt wird.
  
  Äquilibrierung von Zeilen \textbf{und} Spalten führt zu einem nichtlinearen Gleichungssystem und ist i.d.R. aufwendig.
  
  
  \subsection{Lemma} \label{4.3.1}
  Sei $A$ zeilenäquilibriert bzgl. der $l_1$-Norm, dann gilt:
  \begin{gather}
  	cond_{\infty}(A) \leq cond_{\infty}(DA)  \label{IV.1.3}
  \end{gather}
  für alle regulären Diagonalmatrizen $D$.
  
  \paragraph{Beweis} siehe Übungsaufgabe

Wie in Kapitel \ref{3} gesehen, kann die Näherungslösung $\widetilde{x}$ 
trotz Pivotisierung und Äquilibrierung noch sehr ungenau sein.


\subsection{Nachiteration} \index{Nachiteration}
Die Näherung $\widetilde{x}$ kann durch Nachiteration verbessert werden. \\
Falls $\widetilde{x}$ exakt ist, gilt:
\begin{gather}
	r(\widetilde{x}) \coloneqq b-A\widetilde{x} =0 \label{IV.1.4}
\end{gather}
ansonsten ist $A(x-\widetilde{x})=r(\widetilde{x}).$ \ \
Also löse die Korrekturgleichung
\begin{gather}
	A\Delta x = r(\widetilde{x}) 	\label{IV.1.5}
\end{gather}
und setze
\begin{gather*}
	x^{(1)} \coloneqq \widetilde{x} +\Delta x
\end{gather*}
Wiederhole dies sooft, bis $x^{(i)}$ \enquote{genau genug} ist.
Die Lösung $\widetilde{x}$ wird durch Nachiteration meist mit sehr gutem Erfolg verbessert
(genaueres in Dahmen/Reusken MISSING (reference))
\eqref{IV.1.5} wird mit der bereits vorhandenen LR-Zerlegung nur mit der neuen rechten Seite $r(\widetilde{x})$ gelöst, d.h. eine vorwärts und eine Rückwärtssubstitution
mit $\mathcal{O}(n^2)$ flops.

\subsection{Bemerkung (nach Skeel 1980)}
Die Gauß-Elimination mit Spaltenpivotsuche und einer Nachiteration ist komponentenweise stabil.


\sectione{Cholesky-Verfahren}
Im Folgenden sei $A$ eine symmetrische, positiv definite Matrix in $\Renn $, d.h.
$A=A^T$ und $\langle x, Ax \rangle = x^TAx > 0 \quad $ für alle $ x\neq 0$. \\
(kurs: \textbf{spd Matrix}) \index{spd Matrix}

\subsection{Satz (Eigenschaften von symm., pos. def. Matrizen)} \label{4.2.1}
Für jede spd Matrix $A\in \Renn $ gilt:
\begin{enumerate}[i)]	
	\item $A$ ist invertierbar
	\item $a_{ii}>0$ für $i=1, \dots , n$
	\item $\max_{ij}|a_{ij}| = \max_{i}a_{ii}$
	\item Bei der Gauß-Elimination ohne Pivotsuche ist jede Restmatrix wieder eine spd Matrix.
\end{enumerate}

\paragraph{Beweis}
\begin{enumerate}[i)]
	\item folgt aus \eqref{IV.2.1}
	\item Sei $e_i$ der i-te Einheitsvektor, so folgt $a_{ii} = e_{i}^TAe_i > 0$.
	\item siehe Übungsaufgabe
	\item Es gilt:
			\begin{align*}
				A^{(1)} &\coloneqq A = \begin{pmatrix}
															a_{11} & z^T \\ 
															z			& B^{(1)}
														\end{pmatrix} \\
				A^{(2)}	&\coloneqq L_1 A^{(1)} 
								= \begin{pmatrix}
											1 & 0 & \dotsm & 0 \\ \\
											-\frac{z}{a_{ii}} && I \\ ~
										\end{pmatrix} 
								= \begin{pmatrix}
											a_{11} &  & z^T & ~ \\ 
											0 \\
											\vdots && B^{(2)} \\ 
											0
										\end{pmatrix} \\
				\Rightarrow L_1A^{(1)}L_1^T  
				&= \begin{pmatrix}
							a_{11} &  & z^T & ~ \\ 
							0 \\
							\vdots && B^{(2)} \\ 
							0
						\end{pmatrix} 
						\cdot  \begin{pmatrix}
										1 &  &	-\frac{z}{a_{11}} & ~ \\ 
										0 \\\nocite{*}
										\vdots && I \\ 
										0
									\end{pmatrix}\\
				&= \begin{pmatrix}
							a_{11} & 0 & \cdots & 0\\ 
							0 \\
							\vdots && B^{(2)} \\ 
							0
						\end{pmatrix} 
			\end{align*}
			Weiterhin gilt:
			\begin{gather*}
				x\neq 0 \Leftrightarrow L_1 x\neq 0 \,
			\end{gather*}
			da $L_1$ invertierbar. Also gilt insgesamt:
			\begin{align*}
				\widetilde{x}^TB^{(2)} \widetilde{x} &= x^T L_1A^{(1)}L_1^Tx
					 &&  \text{für } x\coloneqq \begin{pmatrix}	0 \\ \widetilde{x}\end{pmatrix}\\
				&= (L_1^Tx)^TA(L_1^Tx) > 0
					&& \forall \widetilde{x}\neq 0 
			\end{align*}
			und damit ist auch $B^{(2)}$ spd.
			
			Induktiv folgt hiermit iv).\hfill $\square$
			Insbesondere ergibt sich: 
			\begin{gather*}
				(L_{n-1}\cdot \cdots\cdot L_1)A^{(1)}(L_1^T\cdot \cdots \cdot L_{n-1}^T) 
					= \begin{pmatrix} d_1 & & 0 \\ &\vdots& \\ 0&& d_n\end{pmatrix} \, ,
			\end{gather*}
			wobei $d_i$ das i-te Diagonalelement von $A^{(i)}$ ist und somit $d_i>0$ für $ i= 1, \cdots , n$ gilt. \\
			
			Sei $L\coloneqq (L_1^{-1}\cdot \cdots \cdot L_{n-1}^{-1})$ wie in \eqref{II.1.8}, so ergibt sich:
		\end{enumerate}
		
		
		\subsection{Folgerung} \label{4.2.2}
		Für jede spd Matrix $A$ existiert eine eindeutige Zerlegung der Form 
			\begin{gather*}
				A= LDL^T
			\end{gather*}
		wobei $L$ eine reelle unipotente (d.h. $l_{ii}=1$) \index{unipotent} (, normierte)  untere 
		Dreiecksmatrix  und $D$ eine positive Diagonalmatrix ist. 
		Diese Zerlegung heißt \textbf{rationale Cholesky-Zerlegung}. Die Zerlegung
		\begin{gather}
			A= \bar{L}\bar{L}^T 
%			= 
%			\begin{pmatrix}
%				&&0 \\
%				\vdots & \ddots&\\
%				\dotsm & \dotsm &
%			\end{pmatrix}
\label{IV.2.2}
		\end{gather}
		mit der reellen unteren Dreiecksmatrix
		\begin{gather*}
			\bar{L} = L \begin{pmatrix}
			\sqrt{d_1} &&0 \\
			& \ddots & \\
			0&& \sqrt{d_n}
			\end{pmatrix} = LD^{\frac{1}{2}}
		\end{gather*}
		heißt \textbf{Cholesky-Zerlegung.} \index{Cholesky-Zerlegung}.
		
		Wegen \eqref{IV.2.2} gilt: 
		\begin{align}
			a_{kk} &= \bar{l}_{k1}^{2} + \cdots +  \bar{l}_{kk}^2  \label{IV.2.3} \\
			a_{ik} &= \bar{l}_{i1} \bar{l}_{k1} + \cdots + \bar{l}_{ik} \bar{l}_{kk}  \label{IV.2.4} \\
			IMAGE~MISSING
		\end{align}
		Demnach funktioniert spaltenweises und zeilenweises Berechnen. \\
		
		Es ergibt sich folgender Algorithmus:
		
		
		\subsection{Cholesky-Zerlegung}\index{Cholesky-Zerlegung}
			Der Algorithmus der Cholesky-Zerlegung ist wie folgt:
			
				\begin{pseudocode}{0.55\linewidth}
						for  $k=1, \cdots , n$\\
									~|\> $l_{kk} = (a_{kk}-\sum_{j=1}^{k-1}l_{kj})^{\frac{1}{2}}$ \\
									~|\> for $i= k+1, \cdots , n$ \\
									~|\>~|\> $l_{ik} = ( a_{ik}- \sum_{j=1}^{k-1}l_{ij} l_{kj})/{l_{kk}}$  \\
									~|\>end\\
						end
					\end{pseudocode}
			
			
			
	\subsection{Rechenaufwand in flops}
	Es sind je 
		\begin{enumerate}
			\item[] $\frac{1}{6}(n^2-n) $ Additionen sowie Multiplikationen und 
			\item[]  $\frac{1}{6}(3n^2-3n) $ Divisionen 
		\end{enumerate}
		also ca. $\frac{2}{3} n^2$ flops für große $n$ notwendig. \\
		Im Vergleich zur LR-Zerlegung halbiert sich in etwa der Aufwand.
		
		\subsection{Bemerkung}
		\begin{enumerate}[a)]
			\item Wegen \eqref{IV.2.3} gilt $|\bar{l}_{kj}| \leq \sqrt{a_{kk}}$,
						d.h. die Matrizeneinträge können nicht zu groß werden.
			\item Für spd Matrizen ist der Cholesky-Algorithmus stabil nach \eqref{III.3.13}
			\item Da $A$ symmetrisch ist, muss nur die untere Dreiecksmatrix gespeichert werden.
				       In Algorithmen kann $\bar{L}$ in eine Kopie dieser Dreiecksmatrix geschrieben werden.
			\item Fast singuläre Matrizen können durch die Diagonale erkannt werden.
		\end{enumerate}
		
	\nocite{*}

		
%----------------------------------------------------------------------------------------------
%BACKMATTER
%----------------------------------------------------------------------------------------------
%\backmatter		%for book only, part for index etc.


\printindex		%only with package makeidx
%\listoffigures		
%\listoftables

\printbibliography	%only with package biblatex


\end{document}
%**********************************************************************************************









